\documentclass{beamer}
\usepackage{../macros}
\usetheme{Berlin}
\usecolortheme{beaver}
\title{Introduction aux dérivées}
\author{K.~Nguyen}
\begin{document}

\maketitle

\section{Introduction}

\subsection{Structure}

\begin{frame}
    {Structure de l'exposé}

    Deux parties:
    \begin{enumerate}
        \item Contexte historique,
            importance dans l'histoire des sciences,
            importance aujourd'hui dans les applications.
            \pause{}
        \item Toute la théorie du chapitre.
    \end{enumerate}
\end{frame}

\subsection{Importance du concept de dérivée}

\begin{frame}
    {Rappel: fonctions}

    \begin{exampleblock}
        {Définition --- fonction}

        Les fonctions généralisent les nombres.
        Elles modélisent une grandeur variant en fonction d'un paramètre.
    \end{exampleblock}

    \pause{}
    \begin{exampleblock}
        {Exemples}

        \begin{itemize}
            \item Prix d'un aliment en fonction du poids.
                \pause{}
            \item Prix d'une course de taxi en fonction de la distance.
                \pause{}
            \item Position en fonction du temps.
        \end{itemize}
    \end{exampleblock}
\end{frame}

\begin{frame}
    {Questions}

    \begin{exampleblock}
        {Questions}

        \begin{enumerate}
            \item Comment étudier/décrire la variation de quantités?
                \pause{}
            \item Où sont les extremas d'une fonction?
                \pause{}
            \item Comment la calculatrice évalue-t-elle $\sin \alpha$, $\log x$?
        \end{enumerate}
    \end{exampleblock}
\end{frame}

\begin{frame}
    {Variations et lois scientifiques}

    Comment écrire mathématiquement ces lois?

    \begin{enumerate}
        \item Le PIB varie proportionnellement en fonction de lui-même.
            \pause{}
        \item Au plus la quantité d'atomes de carbone 14 est importante, au plus cette quantité décroît rapidement.
            \pause{}
        \item Au plus la température de l'eau est loin de la température ambiante, au plus elle décroît rapidement.
    \end{enumerate}

    \pause{}
    \begin{alertblock}
        {Constat}

        Il nous manque un ingrédient crucial avant que les mathématiques aient le pouvoir de \textbf{modéliser} et \textbf{prédire}.
    \end{alertblock}
\end{frame}

\begin{frame}
    {Réflexion sur les lois de Newton}

    Une loi universelle ne peut explicitement décrire une quantité,
    mais seulement ses variations.

    \begin{alertblock}
        {Deuxième loi de Newton}
        \begin{align*}
            \vec F = m \vec a
        \end{align*}
    \end{alertblock}
    \pause{}

    \begin{exampleblock}
        {Explication}

        L'accéleration est la \textbf{variation} de la vitesse,
        qui est la \textbf{variation} de la position.
    \end{exampleblock}
\end{frame}

\subsection{Un peu d'histoire}

\begin{frame}
    {Un peu d'histoire}

    \begin{enumerate}
        \item 1637: (de Fermat) méthode des tangentes pour trouver les extrémas.
            \pause{}
        \item 1665: Newton --- Peste noire à Cambridge\\
            \begin{itemize}
                \item 1666: étudie et développe les travaux de Fermat,
                    travaille sur l'optique et la gravitation universelle.
                \item 1669: Rédaction d'un compte rendu intitulé ``méthode des fluxions'' --- naissance de l'analyse mathématique.
            \end{itemize}
            \pause{}
        \item 1675--1677: Travaux de Leibniz (meilleures notations etc.).
            \pause{}
        \item 1687: Philosophiæ Naturalis Principia Mathematica\\
            \begin{itemize}
                \item Début de la \textbf{mathématisation} de la physique.
                \item Réunion mécanique terrestre/céleste, début de la révolution scientifique
                \item Plus grande oeuvre scientifique de tous les temps?
            \end{itemize}
    \end{enumerate}
\end{frame}

\begin{frame}
    {Principia}
    \begin{exampleblock}
        {Citation --- A.~Einstein}

        \begin{quote}
            ``In order to put his system into mathematical form at all,
            Newton had to devise the concept of differential quotients
            and propound the laws of motion in the form of total differential equations
            —perhaps \textbf{the greatest advance in thought that a single individual was ever privileged to make}.''
        \end{quote}
    \end{exampleblock}
\end{frame}

\begin{frame}
    {Déterminisme}

    Les travaux mathématiques sur la dérivée et les équations de Newton
    ont eu une importance considérable sur la philosophie.
    \begin{exampleblock}
        {Théorème --- Déterminisme}

        Les équations de Newton sont entièrement solubles,
        et une connaissance parfaite du présent permet de connaître le passé et de prédire l'avenir.
    \end{exampleblock}
\end{frame}

\begin{frame}
    {Remarque}

    Les mathématiques ont maintenant le pouvoir de \textbf{modéliser} et \textbf{prédire}.
    \pause{}

    Les dérivées sont le langage naturel pour décrire le monde (quantitativement).
\end{frame}

\section{Les dérivées}

\begin{frame}
    {TL;DR}

    La dérivée est un processus liant une quantité à sa \emph{variation}.
    \pause{}

    \begin{center}
        \begin{tabular}
            {|l|l|}

            \toprule
            quantité & quantité \emph{dérivée}\\
            \midrule

            hauteur & croissance\\
            croissance & concavité\\
            déplacement & vitesse\\
            vitesse & accélération\\
            masse & masse volumique\\
            énergie potentielle & force\\
            travail & puissance\\
            \bottomrule
        \end{tabular}
    \end{center}
\end{frame}

\begin{frame}
    {Dérivée}

    \begin{exampleblock}
        {Question introductive}

        Imaginons que l'on vous bande les yeux et que l'on vous dépose au Stimont.

        \pause{}
        Comment savoir si vous êtes en montée ou en descente?
    \end{exampleblock}
\end{frame}

\begin{frame}
    {Dérivée: définition}

    Imaginons que l'on marche sur la courbe d'un graphe, de gauche à droite.

    \begin{exampleblock}
        {Définition --- dérivée}

        La dérivée est un nombre décrivant l'inclinaison de votre pied lorque vous marchez sur le graphe.
    \end{exampleblock}

    \begin{plot}{0.25}{-10}{-2}{10}{10}
        \plotfunction{-10:10}{-0.09*(\x)^2 + 6}
        \plotfunction{-1:3}{-0.18*(\x - 1) + 6 - 0.09}
    \end{plot}
\end{frame}

\begin{frame}
    {Dérivée: signification}

    Interprétation du nombre dérivée:
    \begin{itemize}
        \item Si la dérivée est \textbf{très positive}: pied très incliné vers le \textbf{haut};
            \pause{}
        \item Si la dérivée est \textbf{positive}: pied incliné vers le \textbf{haut};
            \pause{}
        \item Si la dérivée est \textbf{nulle}: pied \textbf{plat};
            \pause{}
        \item Si la dérivée est \textbf{négative}: pied incliné vers le \textbf{bas};
            \pause{}
        \item Si la dérivée est \textbf{très négative}: pied très incliné vers le \textbf{bas};
            \pause{}
    \end{itemize}
\end{frame}

\begin{frame}
    {Fonction dérivée}

    \begin{exampleblock}
        {Définition: fonction dérivée}

        La fonction dérivée $f'$ de $f$ est la fonction qui à chaque point
        associe la pente du pied en ce point lorsque l'on marche sur le graphe
    \end{exampleblock}
    \pause{}

    \begin{plot}{0.25}{-10}{-2}{10}{10}
        \plotfunction{-10:10}{0.09*(\x)^2}
        \plotfunction{-10:10}{0.18*(\x)}
    \end{plot}
    Si l'on considère la position en fonction du temps,
    la dérivée est la vitesse.
\end{frame}

\begin{frame}
    {Dérivabilité}

    Parfois,
    on ne peut pas poser son pied sur le graphe sans tomber\footnote{Merci Mme Deroeux pour la suggestion}.
    \begin{plot}{0.25}{-10}{-1}{10}{10}
        \plotfunction{-10:10}{abs(\x)}
        \plotfunction{0:10}{5*sqrt(\x)}
    \end{plot}
    \pause{}

    \begin{exampleblock}
        {Définition -- dérivabilité}

        Une fonction $f$ est dérivable en un point si l'on peut poser son pied à cet endroit sur le graphe sans tomber.
    \end{exampleblock}
\end{frame}

\begin{frame}
    {Théorème de Fermat}

    \begin{exampleblock}
        {Théorème de Fermat}

        Si la montagne est lisse,
        alors un sommet se repère par un changement d'inclinaison du pied.
    \end{exampleblock}
    \pause{}

    \begin{alertblock}
        {Théorème de Fermat}
        Si la fonction est dérivable,
        alors un extremum se repère par un changement de signe de la dérivée.
    \end{alertblock}
\end{frame}

\begin{frame}
    {Croissance et signe de la dérivée}

    \begin{exampleblock}
        {Théorème}

        Sur une montagne lisse, une section est une montée (resp.\ descente) ssi votre pied est plat ou tourné vers le haut (resp.\ bas) sur toute cette section.
    \end{exampleblock}
    \pause{}

    \begin{alertblock}
        {Théorème}

        Si une fonction est dérivable, elle est croissante (resp.\ décroissante) sur un intervalle ssi la dérivée est positive (resp.\ négative) sur cet intervalle.
    \end{alertblock}

    \pause{}
    En termes de vitesse: un véhicule avance (resp.\ recule) ssi sa vitesse est positive (resp.\ négative).
\end{frame}

\begin{frame}
    {Variations: vitesses, débit}

    Nous avons principalement parlé de variation en termes de \textbf{croissance}.

    La dérivée modélise en fait également
    \begin{enumerate}
        \item les variations de la position: vitesse, accélération, etc.
            \pause{}
        \item des débits.
            \pause{}
        \item des densités de masse, probabilité, etc.
            \pause{}
        \item des trucs en économie dont j'ai oublié le nom.
    \end{enumerate}
\end{frame}

\begin{frame}
    {Applications des dérivées}

    \begin{enumerate}
        \item Écrire des équations modélisant la réalité.
            \pause{}
        \item Étude de variations d'une fonction.
            \pause{}
        \item Approximations polynômiales
            \begin{enumerate}
                \item Simplifications de problèmes scientifiques (e.g.\ pendule)
                    \pause{}
                \item Méthode de Newton-Raphson pour trouver des racines.
                    \pause{}
                \item Calculatrice: évaluation de fonctions $\log$, $\sin$, etc.
            \end{enumerate}
            \pause{}
        \item Facilite le calcul d'aire, volume, travail, probabilité, etc.\\
    \end{enumerate}
\end{frame}

\begin{frame}
    {Conclusion}

    \begin{enumerate}
        \item La dérivée étudie est un outil d'étude de \textbf{variations} avec des applications extrêmement vastes.
            \pause{}
        \item La dérivée est le \textbf{langage naturel} des lois scientifiques, économiques, etc.
            \pause{}
        \item Son introduction a complètement boulversé l'histoire des sciences, et même l'histoire
            (Galilée, Newton, Lumières).
    \end{enumerate}
\end{frame}

\begin{frame}
    {Dérivée de la fonction constante}

    Si $f(x) = 3$,
    \begin{plot}
        {0.25}{-6}{-6}{6}{6}
        \plotfunction{-6:6}{3}
    \end{plot}
    \pause{}
    alors $f'(x) = 0$ (pied plat).
\end{frame}

\begin{frame}
    {Dérivée de fonctions affines}

    Si $f(x) = 3x + 2$,
    \begin{plot}
        {0.25}{-6}{-6}{6}{6}
        \plotfunction{-6:6}{3*(\x) + 2}
    \end{plot}
    \pause{}
    alors $f'(x)$ est \emph{constante}.
    \pause{}
    En fait, $f'(x) = 3$ (coefficient devant le $x$).
\end{frame}

\begin{frame}
    {Dérivée de fonctions quadratiques}

    Si $f(x) = 4 x^2 + 3x + 2$,
    \begin{plot}
        {0.25}{-6}{-6}{6}{6}
        \plotfunction{-6:6}{4*(\x)^2 + 3*(\x) + 2}
    \end{plot}
    \pause{}
    alors $f'(x) = 4 \cdot 2 x + 3 = 8 x + 3$.
\end{frame}

\end{document}
