\documentclass[main.tex]{subfiles}
\begin{document}

\chapter{Limites et asymptotes}

\section{Introduction}

% TODO David: À retravailler
De nombreuses notions sont introduites dans le secondaire inférieur par l'intermédiaire d'expressions
qui ne sont valides que dans des cas particuliers.
Pour ne citer qu'elle,
la vitesse est introduite comme étant le quotient d'une mesure de distance et d'un intervalle de temps.
Le cas du mouvement rectiligne uniforme mis à part,
cette définition n'est malheureusement qu'approximative,
bien qu'elle produise des résultats de plus en plus précis
au fur et à mesure que l'intervalle de temps considéré devient petit.
Dans cette optique,
il serait tentant de calculer le quotient sur un intervalle de temps nul,
mais nous sortons malheureusement du domaine de définition du quotient.

Ce problème se manifeste pour de nombreuses quantités physiques.
Pour les traiter toutes simultanément,
nous identifions le problème abstrait sous-jacent suivant:
peut-on généraliser la notion d'image d'un point en dehors du domaine?
C'est pour répondre à cette question fondamentale que le concept de \emph{limite} est introduit.

Les applications des limites sont si vastes
qu'elles ont profondément marqué les mathématiques.
Ainsi, les nombres réels ont été définis pour assurer l'existence des limites.
Par ailleurs,
une volonté d'obtenir des limites plus esthétiques a conduit à l'abandon du degré comme mesure d'angle privilégiée au profit du radian.

Plus simplement,
la notion de limite est fondamentale
car elle constitue le pilier sur lequel repose le \emph{calcul différentiel et intégral}.

\section{Limites en un réel}

\subsection{Adhérence}

\subsection{Définition de limite}

\begin{definition}
    [Limite réelle en un réel]

    Soit $f$ une fonction.
    Soit $a \in \cl {\dom f}$ et $L \in \R$.
    On écrit
    \begin{align}
        \lim_{x \to a} f(x) = L
    \end{align}
    si \emph{$f(x)$ est aussi proche que l'on veut de $L$}
    lorsque \emph{$x$ est suffisament proche de $a$ avec $x \in \dom f$}.
\end{definition}

\subsection{Interprétation graphique}

\begin{definition}
    [Asymptote verticale]

    Soit $f$ une fonction et $a \in \cl {\dom f}$.
    Si une limite latérale en $a$ a un sens, existe et est infinie,
    alors la droite d'équation $x = a$ est
    une \emph{asymptote verticale} de $f$.
\end{definition}

\section{Limites à l'infini}

\begin{figure}
    \centering
    \begin{plot}{0.25}{-10}{-10}{10}{10}
        \plotfunction{-10:-1}{\x}
        \plotfunction{-0.99:10}{1/(\x + 1) + 1}
        \drawpoint{-1, -1}
        \drawpoint{1, 1.5}
    \end{plot}
    \caption{Graphe d'une fonction $f$}
    Les limites intéressantes sont les suivantes:
    \begin{align}
        &\lim_{x \to -\infty} f(x) = -\infty, \quad
        &\lim_{x \to +\infty} f(x) = 1,\\
        &\lim_{x \to -1^-} f(x) = -1, \quad
        &\lim_{x \to -1^+} f(x) = +\infty\\
        &\lim_{x \to 1} f(x) = 1,5.
    \end{align}
\end{figure}

\begin{definition}
    [Limite infinie en un réel par la droite]

    Soit $f$ une fonction.
    Soit $a \in \cl {\dom f \cap \cointerval{a}{+\infty}}$.
    On écrit
    \begin{align}
        \lim_{x \to a^+} f(x) = +\infty
    \end{align}
    si \emph{$f(x)$ devient supérieur à n'importe quel nombre positif}
    lorsque \emph{$x$ est suffisament proche de $a$ avec $x \in \dom f$ et $x \geq a$}.
\end{definition}

\begin{definition}
    [Limite réelle à l'infini]

    Soit $f$ une fonction.
    Soit $L \in \R$ et supposons que $\dom f$ ne soit pas majoré.
    On écrit
    \begin{align}
        \lim_{x \to +\infty} f(x) = L
    \end{align}
    si \emph{$f(x)$ est aussi proche que l'on veut de $L$}
    lorsque \emph{$x$ est supérieur à un nombre positif suffisament grand avec $x \in \dom f$}.
\end{definition}

\section{Propriétés des limites}

La propriété suivante justifie le fait que nous pouvons parler de \emph{la} limite en un \emph{point adhérent}.
Elle est fausse sans l'hypothèse d'adhérence.

\begin{proposition}
    [Unicité des limites]

    \subsubsection{Hypothèses}
    \begin{itemize}
        \item $f$ est une fonction;
        \item $a \in \cl {\dom f}$;
        \item $f$ admet $L_1, L_2 \in \R$ comme limite en $a$.
    \end{itemize}

    \subsubsection{Thèse}
    On a $L_1 = L_2$.
\end{proposition}

La limite d'une somme est la somme des limites
lorsque cette dernière a un sens.

\begin{proposition}
    [Limite d'une somme ou différence]

    \subsubsection{Hypothèses}
    \begin{itemize}
        \item $f, g$ deux fonctions;
        \item $a \in \cl {\dom f \cap \dom g}$;
        \item $f$ et $g$ ont une limite en $a$.
    \end{itemize}

    \subsubsection{Thèses}
    \begin{itemize}
        \item $f \pm g$ a une limite en $a$;
        \item On a
            \begin{align}
                \lim_{x \to a} (f \pm g)(x) = \lim_{x \to a} f(x) \pm \lim_{x \to a} g(x).
            \end{align}
    \end{itemize}
\end{proposition}

La limite d'un produit est le produit des limites
lorsque ce dernier a un sens.

\begin{proposition}
    [Limite d'un produit]

    \subsubsection{Hypothèses}
    \begin{itemize}
        \item $f, g$ deux fonctions;
        \item $a \in \cl {\dom f \cap \dom g}$;
        \item $f$ et $g$ ont une limite en $a$.
    \end{itemize}

    \subsubsection{Thèses}
    \begin{itemize}
        \item $f g$ a une limite en $a$;
        \item On a
        \begin{align}
            \lim_{x \to a} (fg)(x) = \left(\lim_{x \to a} f(x)\right) \left(\lim_{x \to a} g(x)\right).
        \end{align}
    \end{itemize}
\end{proposition}

La limite d'un quotient est le quotient des limites
lorsque ce dernier a un sens.

\begin{proposition}
    [Limite d'un quotient]

    \subsubsection{Hypothèses}
    \begin{itemize}
        \item $f, g$ deux fonctions;
        \item $a \in \cl {\dom f/g}$;
        \item $f$ et $g$ ont une limite en $a$;
        \item $\lim_{x \to a} g(x) \neq 0$.
    \end{itemize}

    \subsubsection{Thèses}
    \begin{itemize}
        \item $f/g$ a une limite en $a$;
        \item On a
            \begin{align}
                \lim_{x \to a} \left(\frac f g\right)(x) = \frac {\lim_{x \to a} f(x)} {\lim_{x \to a} g(x)}.
            \end{align}
    \end{itemize}
\end{proposition}

\begin{proposition}
    [Limite d'une fonction composée]

    \subsubsection{Hypothèses}
    \begin{itemize}
        \item $g, f$ deux fonctions;
        \item $a \in \cl {\dom f}$;
        \item $\lim_{x \to a} f(x)$ existe et appartient à $\cl {\dom g}$;
        \item $g$ a une limite en $b \defeq \lim_{x \to a} f(x)$;
    \end{itemize}

    \subsubsection{Thèses}
    \begin{itemize}
        \item $g \circ f$ a une limite en $a$;
        \item On a
            \begin{align}
                \lim_{x \to a} (g \circ f)(x) = \lim_{x \to b} g(x).
            \end{align}
    \end{itemize}
\end{proposition}

\section{Calcul de limites}

Dans les cas les plus simples,
le calcul de la limite en un réel consiste simplement à \emph{évaluer} la fonction en ce réel.

Si le calcul de la limite en $a$ d'une fonction $f$ revient à calculer $f(a)$, on dit que la fonction est \emph{continue} en $a$.

\begin{definition}
    [Fonction continue en un réel]
    
    Soit $f$ une fonction et $a \in \dom f$.
    On dit que $f$ est \emph{continue} en $a$ 
    si $\lim_{x \to a} f(x)$ existe.
\end{definition}

Les \emph{fonction usuelles} sont continues en tout point de leur domaine de définition.

\begin{howto}
    [Quelles sont les fonctions usuelles ?]
    
    Les fonctions usuelles sont :
    \begin{itemize}
        \item les polynômes;
        \item les racines $n$\textsuperscript e;
        \item la fonction inverse $f(x) = \frac 1 x$;
        \item la fonction \emph{valeur absolue} $f(x) = |x|$;
        \item les fonctions trigonométriques $\sin$, $\cos$, $\tan$, $\cot$.
    \end{itemize}
\end{howto}

En utilisant les propositions de la sections précédente, on obtient les résultats suivants.

\begin{proposition}

    \subsubsection{Hypothèses}
    \begin{itemize}
        \item $f, g$ deux fonctions;
        \item $a \in \dom f \cap \dom g$;
        \item $f ,g$ continues en $a$.
    \end{itemize}

    \subsubsection{Thèses}
    \begin{itemize}
        \item $f+g$ continue en $a$;
        \item $f-g$ continue en $a$;
        \item $fg$ continue en $a$.
    \end{itemize}
\end{proposition}

\begin{proposition}

    \subsubsection{Hypothèses}
    \begin{itemize}
        \item $f, g$ deux fonctions;
        \item $a \in \dom f \cap \dom g$;
        \item $f ,g$ continues en $a$;
        \item $g(a) \ne 0$.
    \end{itemize}

    \subsubsection{Thèses}
    \begin{itemize}
        \item $f/g$ continue en $a$.
    \end{itemize}
\end{proposition}

\begin{proposition}

    \subsubsection{Hypothèses}
    \begin{itemize}
        \item $g, f$ deux fonctions;
        \item $a \in \dom f$;
        \item $f(a) \in \dom g$;
        \item $f$ continue en $a$;
        \item $g$ continue en $f(a)$.
    \end{itemize}

    \subsubsection{Thèses}
    \begin{itemize}
        \item $g \circ f$ continue en $a$;
    \end{itemize}
\end{proposition}

On se rend compte que presque toutes les fonctions auxquelles on pourrait penser sont continues en tout point de leur domaine de définition.
Calculer les limites de ces fonctions en ces points est donc aisé.

\begin{example}
 
    Calculons $\lim_{x \to 2} x^2 + |x|.$ Les fonctions $g(x) = x^2$ et $h(x) = |x|$ sont continues en $2$ car ce sont des fonction usuelles.
    La fonction $f(x) = g(x) + h(x) = x^2 + |x|$ est continue en $2$ car c'est une somme de fonctions continues en $2$.
    On a donc $\lim_{x \to 2} x^2 + |x| = 2^2 + |2| = 6$.
\end{example}

\begin{example}
 
    Calculons $\lim_{x \to 0} \sin x - \cos x.$ Les fonctions $g(x) = \sin x$ et $h(x) = \cos x$ sont continues en $0$ car ce sont des fonction usuelles.
    La fonction $f(x) = g(x) - h(x) = \sin x - \cos x$ est continue en $0$ car c'est une différence de fonctions continues en $0$.
    On a donc $\lim_{x \to 0} \sin x - \cos x = \sin 0 - \cos 0 = -1$.
\end{example}

\begin{example}
 
    Calculons $\lim_{x \to 8} 4 \sqrt[3]{x}$ Les fonctions $g(x) = 4$ et $h(x) = \sqrt[3]{x}$ sont continues en $8$ car ce sont des fonction usuelles.
    La fonction $f(x) = g(x)h(x) = 4 \sqrt[3]{x}$ est continue en $8$ car c'est un produit de fonctions continues en $8$.
    On a donc $\lim_{x \to } 4 \sqrt[3]{x} = 4 \sqrt[3]{8} = 8$.
\end{example}

\begin{example}
 
    Calculons $\lim_{x \to 4} 1/|x|$ Les fonctions $g(x) = 1$ et $h(x) = |x|$ sont continues en $4$ car ce sont des fonction usuelles.
    La fonction $f(x) = g(x)/h(x) = 1/|x|$ est continue en $4$ car c'est un quotient de fonctions continues en $4$
    et car $h(4) = |4| = 4 \ne 0$.
    On a donc $\lim_{x \to } 1/|x| = 1/|4| = 1/4$.
\end{example}

\begin{example}
 
    Calculons $\lim_{x \to 9} \sqrt{x^3}$. La fonction $g(x) = x^3$ est continue en $9$ car c'est une fonction usuelle. 
    La fonction $h(x) = \sqrt{x}$ est continue en $g(9) = 729$.
    La fonction $f(x) = \sqrt{x^3} = h(g(x)$ est donc continue en $9$.
    On a donc $\lim_{x \to 9} \sqrt{x^3} = \sqrt{9^3} = 27$.
\end{example}

De manière générale, on peut admettre la proposition suivante :

\begin{proposition}

    \subsubsection{Hypothèses}
    \begin{itemize}
        \item $f, g$ deux fonctions polynômes;
        \item $a \in \R$;
        \item $g(a) \ne 0$.
    \end{itemize}

    \subsubsection{Thèses}
    \begin{itemize}
        \item $\lim_{x \to a} f(x)/g(x) = f(a)/g(a)$.
    \end{itemize}
\end{proposition}

\section{Asymptotes}

\begin{definition}
    [Asymptote horizontale]

    Soit $f$ une fonction.
    Si l'une des limites à l'infini a un sens, existe et vaut $b \in \R$,
    alors la droite d'équation $y = b$ est
    une \emph{asymptote horizontale} de $f$.
\end{definition}

\begin{figure}
    \centering
    \begin{plot}{0.5}{-3}{-4}{9}{8}
        \plotfunction{-3:9} {1 / (\x-3) + 2}
        \drawline (3, -4) -- (3, 8);
        \drawline (-3, 2) -- (9, 2);
    \end{plot}
    \caption{Graphe de $f(x) = \frac 1 {x - 3} + 2$ et ses asymptotes.}
\end{figure}

\section{Résumé}

En combinant les lignes du tableau~\ref{table:limit_definition},
on peut déduire chacune des définitions de limites.
Donnons quelques exemples importants de telles définitions.

\begin{definition}
    [Limite]

    Soit $f$ une fonction.
    \textbf{[Hypothèses supplémentaires]}.
    On écrit \textbf{[limite à définir]}
    si \textbf{[comportement des ordonnées]}
    lorsque \textbf{[comportement des abscisses]}.
\end{definition}

\begin{sidewaystable}
    \centering
    \caption{Tableau récapitulatif pour les définitions de limite}
    \label{table:limit_definition}
    \begin{tabular}
        {l l l l}
        \toprule
        Maths & se lit\dots & Hypothèses & En français \\ \midrule
        $\lim f(x) = L$ & $f(x)$ tend vers $L$ & $L \in \R$ & $f(x)$ est aussi proche que l'on veut de $L$ \\
        $\lim f(x) = +\infty$ & $f(x)$ tend vers $+\infty$ & & $f(x)$ devient supérieur à n'importe quel nombre réel positif\\
        $\lim f(x) = -\infty$ & $f(x)$ tend vers $-\infty$ & & $f(x)$ devient inférieur à n'importe quel nombre réel négatif\\
        $x \to a$ & $x$ tend vers $a$ & $a \in \cl {\dom f}$ & $x$ est suffisament proche de $a$, avec $x \in \dom f$\\
        $x \to a^+$ & $x$ tend vers $a$ par la droite & $a \in \cl {\dom f \cap \cointerval{a}{+\infty}}$ & $x$ est suffisament proche de $a$ avec $x \in \dom f$ et $x \geq a$\\
        $x \to a^-$ & $x$ tend vers $a$ par la gauche & $a \in \cl {\dom f \cap \ocinterval{-\infty}{a}}$ & $x$ est suffisament proche de $a$ avec $x \in \dom f$ et $x \leq a$\\
        $x \to +\infty$ & $x$ tend vers $+\infty$ & $\dom f$ non majoré & $x$ est supérieur à un nombre réel positif suffisament grand avec $x \in \dom f$\\
        $x \to -\infty$ & $x$ tend vers $-\infty$ & $\dom f$ non minoré & $x$ est inférieur à un nombre réel négatif suffisament grand en valeur absolue avec $x \in \dom f$\\
        \bottomrule
    \end{tabular}
\end{sidewaystable}

\end{document}
