\documentclass[main.tex]{subfiles}
\begin{document}

\chapter{Dérivées}

\section{Introduction}

L'introduction du concept de dérivée est historiquement importante,
puisqu'elle coincide avec la naissance de l'analyse mathématique et de la mécanique classique de Newton.
Ce année-là\footnote{%
    L'année 1666 est souvent qualifiée de \emph{miraculeuse} (\emph{annus mirabilis} pour les prétentieux)
    car Newton fait des avancées considérables dans l'étude de:
    l'analyse mathématique, la dynamique, l'optique et la gravitation.

    Un autre example d'\emph{annus mirabilis} est l'année 2009 du FC Barcelone.
    Ils ont gagné la Liga, la Coupe, la Super Coupe, la Champions League, la Super Coupe d'Europe et la Coupe du monde des clubs.
},
les mathématiques ont acquis le pouvoir d'exprimer les lois naturelles et de modéliser des situations complexes.

\section{Définition et premières propriétés}

\begin{definition}
    [Dérivée]

    \subsubsection*{Cadre}

    \begin{itemize}
        \item $f$ est une fonction;
        \item $x \in \cl {\dom f}$ est un point \emph{non-isolé} de $\dom f$.
    \end{itemize}

    \subsubsection*{Définition}

    Si la limite
    \begin{align}
        \lim_{h \to 0} \frac {f(x + h) - f(x)} h
    \end{align}
    existe et est réelle,
    alors le nombre
    \begin{align}
        f'(x) \defeq \lim_{h \to 0} \frac {f(x + h) - f(x)} h
    \end{align}
    est appelé la \emph{dérivée} de $f$ en $x$.
\end{definition}

\begin{remark}
    [Dérivabilité]

    \begin{itemize}
        \item Nous dirons qu'une fonction $f$ est \textbf{dérivable en $x$}
            si la dérivée de $f$ en $x$ a un sens et existe.
        \item Nous dirons qu'une fonction $f$ est \textbf{dérivable}
            si elle est dérivable en tout point de son domaine.
    \end{itemize}
\end{remark}

\begin{remark}
    Une expression telle que
    \begin{align}
        \frac {f(x + h) - f(x)} h
    \end{align}
    est appelée \emph{quotient différentiel}.
    Dans ce langage,
    une dérivée est une limite d'un quotient différentiel.
\end{remark}

\begin{example}
    [Dérivée de $x^2$]

    Calculons la dérivée de $f(x) = x^2$ en un $x \in \R$ quelconque.
    Dans ce but,
    analysons le quotient différentiel
    \begin{align}
        \frac {f(x + h) - f(x)} h
        &= \frac {{(x + h)}^2 - x^2} h\\
        &= \frac {x^2 + 2xh + h^2 - x^2} h\\
        &= 2x + h
    \end{align}

    En faisant tendre $h$ vers $0$ des deux côtés,
    on obtient
    \begin{align}
        f'(x) = 2x.
    \end{align}
\end{example}

\begin{example}
    [Dérivée de $\sqrt x$]

    Calculons la dérivée de $f(x) = \sqrt x$ en un $x \in \R^+$ quelconque.
    Dans ce but,
    analysons le quotient différentiel
    \begin{align}
        \frac {f(x + h) - f(x)} h
        &= \frac {\sqrt {x + h} - \sqrt x} h\\
        &= \frac {x + h - x} {h (\sqrt {x + h} + \sqrt {x})}\\
        &= \frac 1 {\sqrt {x + h} + \sqrt {x}}
    \end{align}

    En faisant tendre $h$ vers $0$ des deux côtés,
    on obtient
    \begin{align}
        f'(x) = \frac 1 {2 \sqrt x}
    \end{align}
\end{example}

\begin{proposition}
    [Dérivabilité et continuité]

    Soit $f$ une fonction dérivable en $x \in \R$.
    La limite
    \begin{align}
        \lim_{h \to 0} f(x + h)
    \end{align}
    existe et vaut $f(x)$
    (autrement dit, $f$ est \emph{continue} en $x$).
\end{proposition}
\begin{proof}
    Soit $h \neq 0$.
    On vérifie que
    \begin{align}
        f(x + h) = \left(\frac {f(x + h) - f(x)} h\right) h + f(x).
    \end{align}

    En faisant tendre $h$ vers $0$ des deux côtés\footnote{%
        Techniquement, on fait tendre $h \to 0$ avec la contrainte supplémentaire $h \neq 0$.
        En d'autres termes, on calcule une limite \emph{épointée}.
        Il faut ensuite conclure qu'au vu du résultat obtenu (i.e.\ $f(x)$),
        la contrainte peut être levée.
    }
    et en employant les propriétés des limites,
    on obtient
    \begin{align}
        \lim_{h \to 0} f(x + h)
        &= \left(\lim_{h \to 0} \frac {f(x + h) - f(x)} h\right) \lim_{h \to 0} h + \lim_{h \to 0} f(x)\\
        &= f'(x) \cdot 0 + f(x) = f(x).
    \end{align}
\end{proof}

Le choix de mesurer les angles en radian
(c'est-à-dire une unité où le tour complet est représenté par un nombre \emph{irrationnel}\footnote{%
    Formulé comme ça, ça a quand même l'air d'être une décision vachement stupide.
    En plus, $\pi$ est transcendant, ce qui est le pire type de nombre irrationnel.

    En tant qu'ami de Monsieur Masson (paix à son âme),
    je me vois également moralement obligé de mentionner
    que la valeur de $\pi$ devrait être doublée pour correspondre au tour complet.
})
est justifié par la proposition suivante (admise sans preuve).

\begin{proposition}
    [Dérivée des fonctions trigonométriques en radian]

    Pour tout $x \in \R$,
    nous avons
    \begin{align}
        \sin' x &= \cos x\\
        \cos' x &= -\sin x.
    \end{align}
\end{proposition}

\section{Comportement avec les opérations sur les fonctions}

Une dérivée peut se calculer terme par terme.

\begin{proposition}
    [Dérivée d'une somme/différence]

    Soient $f, g$ deux fonctions dérivables en $x \in \R$;
    Alors $f \pm g$ est dérivable en $x$ et
    \begin{align}
        (f \pm g)'(x) = f'(x) \pm g'(x).
    \end{align}
\end{proposition}
\begin{proof}
    Soit $h \neq 0$.
    Le quotient différentiel de $f \pm g$ s'écrit
    \begin{align}
        \frac {(f \pm g)(x + h) - (f \pm g)(x)} h
        &= \frac {f(x + h) - f(x)} h \pm \frac {g(x + h) - g(x)} h.
    \end{align}

    En faisant tendre $h$ vers $0$ des deux côtés,
    on obtient
    \begin{align}
        (f \pm g)'(x) = f'(x) \pm g'(x).
    \end{align}
\end{proof}

\begin{proposition}
    [Dérivée d'un produit]

    Soient $f, g$ deux fonctions dérivables en $x \in \R$.
    Alors $f g$ est dérivable en $x$ et
    \begin{align}
        (f g)'(x) = f'(x) g(x) + f(x) g'(x).
    \end{align}
\end{proposition}
\begin{proof}
    Soit $h \neq 0$.
    Le quotient différentiel de $f g$ s'écrit
    \begin{align}
        \frac {(f g)(x + h) - (f g)(x)} h
        &= \frac {f(x + h) g(x + h) - f(x) g(x)} h\\
        &= \frac {f(x + h) g(x + h) - f(x) g(x + h) + f(x) g(x + h) - f(x) g(x)} h\\
        &= \left(\frac {f(x + h) - f(x)} h\right) g(x + h) + f(x) \left(\frac {g(x + h) - g(x)} h\right).
    \end{align}

    En faisant tendre $h$ vers $0$ des deux côtés,
    on obtient
    \begin{align}
        (f g)'(x) = f'(x) g(x) + f(x) g'(x).
    \end{align}
\end{proof}

\begin{proposition}
    [Dérivée d'une composée]

    \subsubsection{Hypothèses}

    \begin{enumerate}
        \item $f$ une fonction dérivable en $x$;
        \item $g$ une fonction dérivable en $f(x)$.
    \end{enumerate}

    \subsubsection{Thèse}

    Alors $g \circ f$ est dérivable en $x$ et
    \begin{align}
        (g \circ f)'(x) = g'(f(x)) f'(x).
    \end{align}
\end{proposition}
\begin{proof}[Lorsque $f$ est injective]
    Soit $h \neq 0$.
    Le quotient différentiel de $g \circ f$ s'écrit
    \begin{align}
        \frac {(g \circ f)(x + h) - (g \circ f)(x)} h
        &= \frac {g(f(x + h)) - g(f(x))} h\\
        &= \frac {g(f(x + h)) - g(f(x))} {f(x + h) - f(x)} \frac {f(x + h) - f(x)} h.
    \end{align}
    Notons que l'hypothèse d'injectivité nous assure qu'il n'y a pas de division par $0$.

    En faisant tendre $h$ vers $0$ des deux côtés,
    et en se souvenant qu'alors $f(x + h)$ tend vers $f(x)$ par continuité,
    on obtient
    \begin{align}
        (g \circ f)'(x) = g'(f(x)) f'(x).
    \end{align}
\end{proof}

\end{document}
