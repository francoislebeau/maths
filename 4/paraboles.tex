\documentclass[main.tex]{subfiles}
\begin{document}

\chapter{Paraboles}

\section{Introduction}

La fonction carrée est l'une des fonctions les plus importantes en mathématiques.
Le théorème de Pythagore implique qu'elle fait partie intégrante de notre façon de mesurer les longueurs,
et les lois de Newton montrent qu'elle décrit (approximativement) la trajectoire d'objets sous l'action du champ gravitationnel terrestre.

\begin{proposition}
    Soit $a, b, c \in \R$ avec $a \neq 0$.
    La fonction définie par
    \begin{align}
        f(x) \defeq ax^2 + bx + c,
        \quad x \in \R
        \label{quadratic_equation}
    \end{align}
    peut être réécrite comme une manipulation graphique de la fonction carré.
    Plus précisément, on a
    \begin{align}
        f(x) = a{\left(x + \frac b {2 a}\right)}^2 - {\frac \Delta {4a}}
    \end{align}
    où $\Delta \defeq b^2 - 4ac$.
\end{proposition}
\begin{proof}
    Mettons en évidence $a$ dans~\eqref{quadratic_equation}:
    \begin{align}
        f(x) = \hspace{5cm}
    \end{align}

    En utilisant le fait que
    \begin{align}
        {\left(x + \frac b {2 a}\right)}^2 = \hspace{4cm},
    \end{align}
    on en déduit que
    \begin{align}
        f(x) &= a\left[{\left(x + \frac b {2a}\right)}^2 - \hspace{3cm}\right]\\
             &=
    \end{align}
\end{proof}

\begin{corollary}
    Soit $a, b, c \in \R$ avec $a \neq 0$.
    L'extremum de la fonction définie par
    \begin{align}
        f(x) \defeq ax^2 + bx + c,
        \quad x \in \R
    \end{align}
    se trouve en $(-b/2a, -\Delta/4a)$.
\end{corollary}
\begin{proof}
    L'extremum de la fonction carré est $(0, 0)$.
    Puisque
    \begin{align}
        f(x) = a\left(x + \frac b {2a}\right)^2 - \frac \Delta {4a},
    \end{align}
    il nous suffit de suivre la trace de l'extremum au fil des transformations graphiques.

    \begin{center}
        \begin{tabular}{l{10cm} l{10cm}}
            \toprule
            Transformation & Position de l'extremum\\
            \midrule
            translation en abscisse de \hspace{3cm} & \\
            dilatation en ordonnée de facteur \hspace {1cm} & \\
            translation en ordonnée de \hspace{3cm} & \\
            \bottomrule
        \end{tabular}
    \end{center}
\end{proof}

\end{document}
