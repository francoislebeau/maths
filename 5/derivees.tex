\documentclass[main.tex]{subfiles}
\begin{document}

\tableofcontents

\chapter{Dérivées}

\begin{definition}
    [Dérivée]

    \subsubsection*{Cadre}

    \begin{itemize}
        \item $f$ est une fonction;
        \item $x \in \cl {\dom f}$ est un point \emph{non-isol\'e} de $\dom f$.
    \end{itemize}

    \subsubsection*{Définition}

    Si la limite
    \begin{align}
        \lim_{h \to 0} \frac {f(x + h) - f(x)} h
    \end{align}
    existe et est réelle,
    alors le nombre
    \begin{align}
        f'(x) \defeq \lim_{h \to 0} \frac {f(x + h) - f(x)} h
    \end{align}
    est appelé la \emph{dérivée} de $f$ en $x$.
\end{definition}

\begin{remark}
    Une expression telle que
    \begin{align}
        \frac {f(x + h) - f(x)} h
    \end{align}
    est appelée \emph{quotient différentiel}.
    Dans ce langage,
    une dérivée est une limite d'un quotient différentiel.
\end{remark}

\begin{example}
    [Dérivée de $x^2$]

    Calculons la dérivée de $f(x) = x^2$ en un $x \in \R$ quelconque.
    Dans ce but,
    analysons le quotient différentiel
    \begin{align}
        \frac {f(x + h) - f(x)} h
        &= \frac {{(x + h)}^2 - x^2} h\\
        &= \frac {x^2 + 2xh + h^2 - x^2} h\\
        &= 2x + h
    \end{align}

    En faisant tendre $h$ vers $0$ des deux côtés,
    on obtient
    \begin{align}
        f'(x) = 2x.
    \end{align}
\end{example}

\end{document}
