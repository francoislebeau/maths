\documentclass[main.tex]{subfiles}
\begin{document}

\tableofcontents

\chapter{Limites et asymptotes}

\section{Introduction}

% TODO David: À retravailler
De nombreuses notions sont introduites dans le secondaire inférieur par l'intermédiaire d'expressions
qui ne sont valides que dans des cas particuliers.
Pour ne citer qu'elle,
la vitesse est introduite comme étant le quotient d'une mesure de distance et d'un intervalle de temps.
Le cas du mouvement rectiligne uniforme mis à part,
cette définition n'est malheureusement qu'approximative,
bien qu'elle produise des résultats de plus en plus précis
au fur et à mesure que l'intervalle de temps considéré devient petit.
Dans cette optique,
il serait tentant de calculer le quotient sur un intervalle de temps nul,
mais nous sortons malheureusement du domaine de définition du quotient.

Ce problème se manifeste pour de nombreuses quantités physiques.
Pour les traiter toutes simultanément,
nous identifions le problème abstrait sous-jacent suivant:
peut-on généraliser la notion d'image d'un point en dehors du domaine?
C'est pour répondre à cette question fondamentale que le concept de \emph{limite} est introduit.

Les applications des limites sont si vastes
qu'elles ont profondément marqué les mathématiques.
Ainsi, les nombres réels ont été définis pour assurer l'existence des limites.
Par ailleurs,
une volonté d'obtenir des limites plus esthétiques a conduit à l'abandon du degré comme mesure d'angle privilégiée au profit du radian.

Plus simplement,
la notion de limite est fondamentale
car elle constitue le pilier sur lequel repose le \emph{calcul différentiel et intégral}.

\section{Limites en un réel}

\subsection{Adhérence}

Un point \emph{adhérent} à un ensemble est un \emph{réel} qui appartient ou ``colle'' à cet ensemble.
L'ensemble des points adhérents à un ensemble $A \subset \R$
s'appelle l'\emph{adhérence de $A$} et se dénote $\cl A$.

Dans le cadre d'un cours du secondaire,
l'adhérence se calcule facilement:
\begin{howto}
    [Calcul d'adhérence]

    \begin{enumerate}
        \item On réecrit l'ensemble comme union d'intervalles.
        \item L'adhérence contient en plus les bornes \emph{réelles} de chacun des intervalles.
    \end{enumerate}
\end{howto}

En pratique,
nous calculons l'adhérence du domaine d'une fonction.

\begin{example}
    [Adhérence d'un domaine]

    Calculons l'adhérence du domaine de la fonction définie par
    \begin{align}
        f(x) = \sqrt{\frac {x + 3}{x - 10}}.
    \end{align}
    On vérifie que
    \begin{align}
        \dom f = \ocinterval {-\infty} {-3} \cup \oointerval {10} {+\infty},
    \end{align}
    et donc on en conclut que
    \begin{align}
        \cl {\dom f} = \ocinterval {-\infty} {-3} \cup \cointerval {10} {+\infty}.
    \end{align}
\end{example}

Du point de vue du calcul des limites,
les réels les plus intéressants sont les \textbf{points adhérents hors du domaine}.

\subsection{Limite réelle en un réel}

\begin{definition}
    [Limite réelle en un réel]

    \subsubsection*{Cadre}
    \begin{enumerate}
        \item $f$ une fonction;
        \item $a \in \cl {\dom f}$;
        \item $L \in \R$.
    \end{enumerate}

    \subsubsection*{Définition}
    On écrit
    \begin{align}
        \lim_{x \to a} f(x) = L
    \end{align}
    si \textbf{$f(x)$ est aussi proche que l'on veut de $L$}
    lorsque \textbf{$x$ est suffisament proche de $a$ avec $x \in \dom f$}.
\end{definition}

\begin{example}
    [Limite en un réel]

    Supposons que nous considérons la fonction $f$ dont le graphe est esquissé ci-dessous.

    \begin{center}
        \begin{plot}{0.5}{-6}{-6}{6}{6}
            \plotfunction{-6:6}{0.25*\x^3 - 0.8*\x^2 - \x - 1}
            \drawpoint{0, -1}
        \end{plot}
    \end{center}

    Nous remarquons que les images deviennent aussi proches que l'on veut de $-1$
    lorsque les abscisses sont suffisament proches de $0$.
    Nous pouvons écrire
    \begin{align}
        \lim_{x \to 0} f(x) = -1.
    \end{align}
\end{example}

Il faut faire très attention
lorsque une limite est calculée à l'intérieur du domaine.
Nous rompons consciemment avec la plupart des ouvrages mathématiques
en disant que la limite n'existe pas dans l'exemple suivant.

\begin{example}
    [Limite pointée]

    Supposons que nous considérons la fonction $f$ dont le graphe est esquissé ci-dessous.

    \begin{center}
        \begin{plot}{0.5}{-6}{-6}{6}{6}
            \plotfunction{-6:6}{0.25*\x^3 - 0.8*\x^2 - \x - 1}
            \drawpoint{0, -1}
            \drawpoint[black]{0, 2}
        \end{plot}
    \end{center}

    Nous remarquons que les images ne sont pas systématiquement proches de $-1$
    lorsque l'on considère des $x$ proches de $0$.
    En particulier, $x = 0$ est proche de $0$
    alors que son image vaut $2$ qui est loin de $-1$.
    Dès lors,
    la limite $\lim_{x \to 0} f(x)$ n'existe pas.
\end{example}

Je sens déjà que je vais me faire traiter d'incompétent par vos profs particuliers,
donc s'il vous plaît défendez-moi et montrez-leur ceci:

\begin{remark}
    [Limite épointée]

    Notre définition coïncide avec la définition classique de limite en dehors du domaine,
    mais peut être différente à l'intérieur.
    À quoi ressemble un cours où la définition de limite inclus le contre-exemple ci-dessus?
    Pour être mathématiquement correct
    (lire: ce que l'Actimath devrait faire),
    un tel cours doit vérifier les points suivants.
    \begin{enumerate}
        \item Toutes les propriétés et définitions doivent être énoncées
            en un réel adhérent et \textbf{non isolé} du domaine.
            Cette hypothèse supplémentaire rallonge littéralement tous les énoncés.
        \item La définition de limite doit être rallongée en excluant explicitement $x = a$.
        \item La définition de \emph{continuité} est plus longue.
        \item La propriété de \emph{composition des limites} serait fausse dans un tel cours sans hypothèse de continuité.
        \item Une personne normale ne calcule que les limites à l'extérieur du domaine, donc on s'en fout.
    \end{enumerate}
\end{remark}

\subsection{Limite infinie en un réel}

\begin{definition}
    [Limite infinie en un réel]

    \subsubsection*{Cadre}
    \begin{enumerate}
        \item $f$ une fonction;
        \item $a \in \cl {\dom f}$;
    \end{enumerate}

    \subsubsection*{Définition}
    On écrit
    \begin{align}
        \lim_{x \to a} f(x) = +\infty \quad
        \left(\text{resp.} \lim_{x \to a} f(x) = -\infty\right)
    \end{align}
    si \textbf{$f(x)$ devient supérieur (resp.\ inférieur) à n'importe quel nombre réel}
    lorsque \textbf{$x$ est suffisament proche de $a$ avec $x \in \dom f$}.
\end{definition}

\begin{example}
    [Limite infinie en un réel]

    Supposons que nous considérons la fonction $f$ dont le graphe est esquissé ci-dessous.

    \begin{center}
        \begin{plot}{0.5}{-6}{-1}{6}{12}
            \plotfunction{-6:6}{1/(\x - 1)^2}
        \end{plot}
    \end{center}

    Nous remarquons que les images deviennent aussi grande que l'on souhaite
    lorsque les abscisses sont suffisament proches de $1$.
    Nous pouvons écrire
    \begin{align}
        \lim_{x \to 1} f(x) = +\infty.
    \end{align}
\end{example}

\subsection{Limites latérales}

\begin{example}
    [Fonctions homographiques]

    Il est clair que la fonction $f$ définie par
    \begin{align}
        f(x) = \frac 1 {x + 1},
    \end{align}
    dont le graphe est ci-dessous,
    \begin{center}
        \begin{plot}{0.5}{-6}{-6}{6}{6}
            \plotfunction{-6:0.99}{1/(\x - 1)}
            \plotfunction{1.01:6}{1/(\x - 1)}
        \end{plot}
    \end{center}
    n'a pas de limite, même infinie, en $1$.

    Nous serions tenté de dire que la limite en $1$ par la gauche est $-\infty$,
    tandis que la limite en $1$ par la droite est $+\infty$.
    Nous noterons ça:
    \begin{align}
        \lim_{x \to 1^-} f(x) = -\infty
        \quad
        \lim_{x \to 1^+} f(x) = +\infty
    \end{align}
\end{example}

\subsection{Interprétation graphique}

Trois cas se présentent lorsqu'une limite existe en un réel.

\begin{enumerate}
    \item Si la limite existe en $a$ et est réelle,
        et que $a \in \dom f$,
        alors $(a, f(a))$ est un \textbf{point plein}.
        Par exemple,
        dans le graphe ci-dessous,
        \begin{center}
            \begin{plot}{0.5}{-6}{-2}{6}{2}
                \plotfunction{-6:6}{0.05*\x^3}
                \drawpoint[black]{0, 0}
            \end{plot}
        \end{center}
        on a $\lim_{x \to 0} f(x) = 0$ donc $(0, 0)$ est un \emph{point plein}.
    \item Si la limite existe en $a$ et est réelle,
        et que $a \notin \dom f$,
        alors $(a, \lim_{x \to a} f(x))$ est un \textbf{point creux}.
        Par exemple,
        dans le graphe ci-dessous,
        \begin{center}
            \begin{plot}{0.5}{-6}{-2}{6}{2}
                \plotfunction{-6:6}{0.05*\x^3}
                \drawpoint{0, 0}
            \end{plot}
        \end{center}
        on a $\lim_{x \to 0} f(x) = 0$ donc $(0, 0)$ est un \emph{point creux}.
    \item Si une limite latérale en $a$ est infinie,
        alors nous avons une \textbf{asymptote verticale} en $a$.
        Intuitivement,
        on voit que proche de $a$,
        le graphe de $f$ se rapproche de la droite d'équation
        \begin{align}
            AV_a \equiv x = a.
        \end{align}
        Dans l'exemple suivant,
        \begin{center}
            \begin{plot}{0.25}{-6}{-1}{6}{12}
                \plotfunction{-6:6}{1/(\x - 1)^2 + 2}
                \drawline (1, -1) -- (1, 13);
            \end{plot}
        \end{center}
        on remarque que $\lim_{x \to 1} = +\infty$,
        et qu'autour de $1$,
        le graphe est proche de l'\emph{asymptote verticale} d'équation
        \begin{align}
            AV_1 \equiv x = 1.
        \end{align}
\end{enumerate}

\begin{definition}
    [Asymptote verticale]

    Soit $f$ une fonction.
    Si une limite latérale de $f$ en $a$ a un sens, existe et est infinie,
    alors la droite d'équation
    \begin{align}
        AV_a \equiv x = a
    \end{align}
    est une \emph{asymptote verticale} de $f$.
\end{definition}

\section{Limites à l'infini}

\begin{figure}
    \centering
    \begin{plot}{0.25}{-10}{-10}{10}{10}
        \plotfunction{-10:-1}{\x}
        \plotfunction{-0.99:10}{1/(\x + 1) + 1}
        \drawpoint{-1, -1}
        \drawpoint{1, 1.5}
    \end{plot}
    \caption{Graphe d'une fonction $f$}
    Les limites intéressantes sont les suivantes:
    \begin{align}
        &\lim_{x \to -\infty} f(x) = -\infty, \quad
        &\lim_{x \to +\infty} f(x) = 1,\\
        &\lim_{x \to -1^-} f(x) = -1, \quad
        &\lim_{x \to -1^+} f(x) = +\infty\\
        &\lim_{x \to 1} f(x) = 1,5.
    \end{align}
\end{figure}

\begin{definition}
    [Limite infinie en un réel par la droite]

    Soit $f$ une fonction.
    Soit $a \in \cl {\dom f \cap \cointerval{a}{+\infty}}$.
    On écrit
    \begin{align}
        \lim_{x \to a^+} f(x) = +\infty
    \end{align}
    si \emph{$f(x)$ devient supérieur à n'importe quel nombre positif}
    lorsque \emph{$x$ est suffisament proche de $a$ avec $x \in \dom f$ et $x \geq a$}.
\end{definition}

\begin{definition}
    [Limite réelle à l'infini]

    Soit $f$ une fonction.
    Soit $L \in \R$ et supposons que $\dom f$ ne soit pas majoré.
    On écrit
    \begin{align}
        \lim_{x \to +\infty} f(x) = L
    \end{align}
    si \emph{$f(x)$ est aussi proche que l'on veut de $L$}
    lorsque \emph{$x$ est supérieur à un nombre positif suffisament grand avec $x \in \dom f$}.
\end{definition}

\section{Propriétés des limites}

La propriété suivante justifie le fait que nous pouvons parler de \emph{la} limite en un \emph{point adhérent}.
Elle est fausse sans l'hypothèse d'adhérence.

\begin{proposition}
    [Unicité des limites]

    \subsubsection*{Hypothèses}
    \begin{itemize}
        \item $f$ est une fonction;
        \item $a \in \cl {\dom f}$;
        \item $f$ admet $L_1, L_2 \in \R$ comme limite en $a$.
    \end{itemize}

    \subsubsection*{Thèse}
    On a $L_1 = L_2$.
\end{proposition}

La limite d'une somme est la somme des limites
lorsque cette dernière a un sens.

\begin{proposition}
    [Limite d'une somme ou différence]

    \subsubsection*{Hypothèses}
    \begin{itemize}
        \item $f, g$ deux fonctions;
        \item $a \in \cl {\dom f \cap \dom g}$;
        \item $f$ et $g$ ont une limite en $a$.
    \end{itemize}

    \subsubsection*{Thèses}
    \begin{itemize}
        \item $f \pm g$ a une limite en $a$;
        \item On a
            \begin{align}
                \lim_{x \to a} (f \pm g)(x) = \lim_{x \to a} f(x) \pm \lim_{x \to a} g(x).
            \end{align}
    \end{itemize}
\end{proposition}

La limite d'un produit est le produit des limites
lorsque ce dernier a un sens.

\begin{proposition}
    [Limite d'un produit]

    \subsubsection*{Hypothèses}
    \begin{itemize}
        \item $f, g$ deux fonctions;
        \item $a \in \cl {\dom f \cap \dom g}$;
        \item $f$ et $g$ ont une limite en $a$.
    \end{itemize}

    \subsubsection*{Thèses}
    \begin{itemize}
        \item $f g$ a une limite en $a$;
        \item On a
        \begin{align}
            \lim_{x \to a} (fg)(x) = \left(\lim_{x \to a} f(x)\right) \left(\lim_{x \to a} g(x)\right).
        \end{align}
    \end{itemize}
\end{proposition}

La limite d'un quotient est le quotient des limites
lorsque ce dernier a un sens.

\begin{proposition}
    [Limite d'un quotient]

    \subsubsection*{Hypothèses}
    \begin{itemize}
        \item $f, g$ deux fonctions;
        \item $a \in \cl {\dom f/g}$;
        \item $f$ et $g$ ont une limite en $a$;
        \item $\lim_{x \to a} g(x) \neq 0$.
    \end{itemize}

    \subsubsection*{Thèses}
    \begin{itemize}
        \item $f/g$ a une limite en $a$;
        \item On a
            \begin{align}
                \lim_{x \to a} \left(\frac f g\right)(x) = \frac {\lim_{x \to a} f(x)} {\lim_{x \to a} g(x)}.
            \end{align}
    \end{itemize}
\end{proposition}

\begin{proposition}
    [Limite d'une fonction composée]

    \subsubsection*{Hypothèses}
    \begin{itemize}
        \item $g, f$ deux fonctions;
        \item $a \in \cl {\dom f}$;
        \item $\lim_{x \to a} f(x)$ existe et appartient à $\cl {\dom g}$;
        \item $g$ a une limite en $b \defeq \lim_{x \to a} f(x)$;
    \end{itemize}

    \subsubsection*{Thèses}
    \begin{itemize}
        \item $g \circ f$ a une limite en $a$;
        \item On a
            \begin{align}
                \lim_{x \to a} (g \circ f)(x) = \lim_{x \to b} g(x).
            \end{align}
    \end{itemize}
\end{proposition}

\section{Calcul de limites}

Dans les cas les plus simples,
le calcul de la limite en un réel consiste simplement à \emph{évaluer} la fonction en ce réel.

Si le calcul de la limite en $a$ d'une fonction $f$ revient à calculer $f(a)$, on dit que la fonction est \emph{continue} en $a$.

\begin{definition}
    [Fonction continue en un réel]

    Soit $f$ une fonction et $a \in \dom f$.
    On dit que $f$ est \emph{continue} en $a$
    si $\lim_{x \to a} f(x)$ existe.
\end{definition}

\section{Asymptotes}

\begin{definition}
    [Asymptote horizontale]

    Soit $f$ une fonction.
    Si l'une des limites à l'infini a un sens, existe et vaut $b \in \R$,
    alors la droite d'équation $y = b$ est
    une \emph{asymptote horizontale} de $f$.
\end{definition}

\begin{figure}
    \centering
    \begin{plot}{0.5}{-3}{-4}{9}{8}
        \plotfunction{-3:9} {1 / (\x-3) + 2}
        \drawline (3, -4) -- (3, 8);
        \drawline (-3, 2) -- (9, 2);
    \end{plot}
    \caption{Graphe de $f(x) = \frac 1 {x - 3} + 2$ et ses asymptotes.}
\end{figure}

\section{Résumé}

En combinant les lignes du tableau~\ref{table:limit_definition},
on peut déduire chacune des définitions de limites.
Donnons quelques exemples importants de telles définitions.

\begin{definition}
    [Limite]

    Soit $f$ une fonction.
    \textbf{[Hypothèses supplémentaires]}.
    On écrit \textbf{[limite à définir]}
    si \textbf{[comportement des ordonnées]}
    lorsque \textbf{[comportement des abscisses]}.
\end{definition}

\begin{sidewaystable}
    \centering
    \caption{Tableau récapitulatif pour les définitions de limite}
    \label{table:limit_definition}
    \begin{tabular}
        {l l l l}
        \toprule
        Maths & se lit\dots & Hypothèses & En français \\ \midrule
        $\lim f(x) = L$ & $f(x)$ tend vers $L$ & $L \in \R$ & $f(x)$ est aussi proche que l'on veut de $L$ \\
        $\lim f(x) = +\infty$ & $f(x)$ tend vers $+\infty$ & & $f(x)$ devient supérieur à n'importe quel nombre réel positif\\
        $\lim f(x) = -\infty$ & $f(x)$ tend vers $-\infty$ & & $f(x)$ devient inférieur à n'importe quel nombre réel négatif\\
        $x \to a$ & $x$ tend vers $a$ & $a \in \cl {\dom f}$ & $x$ est suffisament proche de $a$, avec $x \in \dom f$\\
        $x \to a^+$ & $x$ tend vers $a$ par la droite & $a \in \cl {\dom f \cap \cointerval{a}{+\infty}}$ & $x$ est suffisament proche de $a$ avec $x \in \dom f$ et $x \geq a$\\
        $x \to a^-$ & $x$ tend vers $a$ par la gauche & $a \in \cl {\dom f \cap \ocinterval{-\infty}{a}}$ & $x$ est suffisament proche de $a$ avec $x \in \dom f$ et $x \leq a$\\
        $x \to +\infty$ & $x$ tend vers $+\infty$ & $\dom f$ non majoré & $x$ est supérieur à un nombre réel positif suffisament grand avec $x \in \dom f$\\
        $x \to -\infty$ & $x$ tend vers $-\infty$ & $\dom f$ non minoré & $x$ est inférieur à un nombre réel négatif suffisament grand en valeur absolue avec $x \in \dom f$\\
        \bottomrule
    \end{tabular}
\end{sidewaystable}

\end{document}
