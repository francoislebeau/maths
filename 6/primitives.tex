\documentclass[main.tex]{subfiles}
\begin{document}
\chapter{Primitives}

\section{Théorie}

\begin{definition}
    [Primitive]

    \subsubsection{Cadre}
    \begin{itemize}
        \item $f$ une fonction définie sur un intervalle $I$.
    \end{itemize}

    \subsubsection{Définition}
    $F$ est une \textbf{primitive} de $f$
    si
    \begin{itemize}
        \item $F$ est définie et \emph{dérivable} sur $I$;
        \item $F' = f$ sur $I$.
    \end{itemize}
\end{definition}

\begin{proposition}
    [Primitivation par parties]

    \subsubsection{Hypothèses}
    \begin{itemize}
        \item $f, g$ deux fonctions dérivables sur un intervalle $I$.
    \end{itemize}

    \subsubsection{Thèses}
    \begin{itemize}
        \item $f' g$ est primitivable sur $I$ ssi $f g'$ est primitivable sur $I$.
        \item On a alors
            \begin{align}
                \int f'(x) g(x) \dd x
                = f(x) g(x)
                - \int f(x) g'(x) \dd x.
            \end{align}
    \end{itemize}
\end{proposition}
\begin{proof}
    Partons de la formule de dérivation d'un produit
    \begin{align*}
        (f g)'(x) = f'(x) g(x) + f(x) g'(x).
    \end{align*}

    En réorganisant,
    nous obtenons
    \begin{align*}
        f'(x) g(x) = (f g)'(x) - f(x) g'(x).
    \end{align*}

    En primitivant les deux côtés,
    on obtient
    \begin{align}
        \int f'(x) g(x) \dd x
        &= \int (f g)'(x) \dd x - \int f(x) g'(x) \dd x\\
        &= f(x) g(x) - \int f(x) g'(x) \dd x.
    \end{align}
\end{proof}

\begin{proposition}
    [Changement de variable]

    \subsubsection{Hypothèses}
    \begin{itemize}
        \item $u$ est dérivable sur un intervalle $I$.
        \item $f$ est primitivable sur $u(I)$.
    \end{itemize}

    \subsubsection{Thèses}
    \begin{itemize}
        \item $(f \circ u) u'$ est primitivable sur $I$;
        \item On a
            \begin{align}
                \int f(u(x)) u'(x) \dd x
                = F(u(x)) + C,
            \end{align}
            où $F$ est une primitive de $f$ sur $u(I)$.
    \end{itemize}
\end{proposition}
\begin{proof}
    Soit $F$ une primitive de $f$.
    En partant du membre de droite,
    et en se souvenant que la primitive est l'\emph{inverse} de la dérivée,
    on a
    \begin{align}
        F(u(x)) + C
        &= \int (F(u(x)) + C)' \dd x\\
        &= \int F'(u(x)) u'(x) \dd x,
    \end{align}
    où la deuxième ligne a été obtenue par la dérivée de fonction composée.

    Puisque $F$ est une primitive de $f$, $F' = f$ et donc
    \begin{align}
        F(u(x)) + C
        &= \int f(u(x)) u'(x) \dd x,
    \end{align}
    ce qui était bien le résultat à prouver.
\end{proof}

\section{Techniques de primitivations}

\subsection{Primitives immédiates}

\begin{proposition}
    [Primitives immédiates]

    Soit $p \in \R \setminus \{1\}$.

    \begin{align}
        \int a \dd x &= ax + C &
        \int x^p \dd x &= \frac {x^{p + 1}} {p + 1} + C\\
        \int e^x \dd x &= e^x + C &
        \int a^x \dd x &= \frac {a^x} {\ln a} + C\\
        \int \cos x \dd x &= \sin x + C &
        \int \sin x \dd x &= -\cos x + C\\
        \int \frac 1 {1 + x^2} \dd x &= \arctan x + C.
    \end{align}
    \begin{align}
        \int \frac 1 x \dd x &= \ln \abs x + C &
        \int \frac 1 {\cos^2 x} \dd x &= \tan x + C \\
        \int \frac 1 {\sin^2 x} \dd x &= \cot x + C &
        \int \frac 1 {\sqrt {1 - x^2}} \dd x &= \arcsin x + C.
    \end{align}
\end{proposition}

\subsection{Substitution ou ``changement de variable facile''}

\subsubsection{Quand l'appliquer?}

Statistiquement,
c'est la technique la plus utilisée dans vos exercices (de loin).
À essayer à tout prix.

La question qu'il faut se poser est la suivante:
a-t-on une \textbf{multiplication} par un terme qui ressemble à une dérivée de quelque chose qui apparaît autre part?

Ensuite, suivre les étapes suivantes:
\begin{enumerate}
    \item Choix de $u$;
    \item Calculer la dérivée $u'$;
    \item Écrire la primitive de telle sorte qu'à part exactement une multiplication par $u'$,
        toute dépendance en $x$ s'écrit dépend en fait de $u(x)$.
    \item Remplacer $u$ par $x$ dans un \textbf{calcul intermédiaire à part}.
    \item Résoudre cette primitive différente.
    \item La solution finale est la solution de la primitive plus simple,
        où $x$ est remplacé par $u(x)$.
\end{enumerate}

\begin{example}
    Calculons la primitive
    \begin{align}
        I \defeq \int \frac {x^2} {\cos^2 (x^3)} \dd x.
    \end{align}

    On pose $u(x) = x^3$,
    de telle sorte que $u'(x) = 3x^2$.
    Dès lors, on a
    \begin{align}
        I &= \int \frac {3 x^2} {3 \cos^2 (x^3)} \dd x\\
          &= \frac 1 3 \int \frac {u'(x)} {\cos^2 (u(x))} \dd x.
    \end{align}

    En remplaçant $u(x)$ par $x$,
    on calcule
    \begin{align}
        \int \frac 1 {\cos^2 x} \dd x = \tan x + C.
    \end{align}

    On en conclut (en remplaçant $x$ par $u(x)$ ci-dessus) que
    \begin{align}
        I = \tan (x^3) + C
    \end{align}
\end{example}

\subsection{Primitivation par parties}

\subsubsection{Quand l'appliquer?}

Vraiment en dernier recours,
c'est une technique assez coûteuse.

\begin{enumerate}
    \item Produit entre
        \begin{itemize}
            \item une fonction qui devient ``moins complexe'' après dérivation (e.g.~polynôme)
            \item une fonction facile à primitiver (e.g.~primitive immédiate).
        \end{itemize}
    \item Produit entre fonctions trigonométriques/exponentielles.
    \item Fonction seule n'apparaissant pas dans la liste des primitives immédiates ($\ln$, $\arctan$)
\end{enumerate}

\subsubsection{Cas 1}

On fait le choix suivant:

\begin{itemize}
    \item la fonction qui devient moins complexe après dérivation est la fonction ``normale''.
    \item la fonction facile à primitiver est la fonction ``prime''.
\end{itemize}

\begin{example}
    Cherchons à calculer la primitive suivante:
    \begin{align}
        I \defeq \int x \sin x \dd x.
    \end{align}

    Comme $x$ deviendra plus simple après dérivation,
    nous la choisissons comme fonction normale
    (elle sera dérivée dans le membre de droite)
    et nous posons
    \begin{align}
        f(x) &= x\\
        g'(x) &= \sin x
    \end{align}
    de telle sorte que $g(x) = -\cos x$.

    Par la formule de primitivation par parties,
    nous obtenons
    \begin{align}
        I &= x (-\cos x) - \int 1 (-\cos x) \dd x\\
          &= -x \cos x + \int \cos x \dd x\\
          &= -x \cos x + \sin x + C
    \end{align}
\end{example}

\subsubsection{Cas 2}

Le calcul de ces primitives requiert \textbf{deux applications} de l'intégration par parties.

Le choix est libre,
mais il faut faire le même choix pour les deux applications.

\begin{example}
    J'aurais eu le temps de le taper si les 6G avaient été au lit plus tôt pendant la retraite.
    Voir cours.
\end{example}

\subsubsection{Cas 3}

On choisit $1$ comme fonction dérivée (avec le symbole prime)
et le reste comme fonction ``normale'' (c'est-à-dire \emph{à dériver}).

\begin{example}
    Calculons
    \begin{align}
        I \defeq \int 1 \ln x.
    \end{align}

    Pour cela,
    posons
    \begin{align}
        f'(x) &= 1\\
        g(x) &= \ln x
    \end{align}
    de telle sorte que $f(x) = x$.

    La formule de primitivation par parties donne:
    \begin{align}
        I &= x \ln x - \int x \frac 1 x \dd x\\
          &= x \ln x - x + C.
    \end{align}
\end{example}

\subsection{Changement de variables}

Le changement de variable $u(x)$ sera toujours donné.
Les étapes sont les mêmes que pour la substitution,
mais il faut un peu plus travailler pour se débarasser complètement
de la dépendance explicite en $x$.

Voir feuille manuscrite pour des exemples supplémentaires.

\subsection{Règles du jeu pour la compétence C2}

Vous perdez des points lorsque:

\begin{itemize}
    \item vous oubliez $+ C$ dans chaque membre ne comportant pas de symbole de primitive;
    \item vous faites des erreurs de parenthèses et de priorité des opérations (regardez votre interro précédente);
    \item vous faites des erreurs de calcul (la pénalité sera d'autant plus importante si vous simplifiez l'exercice).
\end{itemize}

\end{document}
