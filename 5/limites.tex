\documentclass[main.tex]{subfiles}
\begin{document}

\chapter{Limites et asymptotes}

\section{Introduction}

% TODO David: À retravailler
De nombreuses notions sont introduites dans le secondaire inférieur par l'intermédiaire d'expressions
qui ne sont valides que dans des cas particuliers.
Pour ne citer qu'elle,
la vitesse est introduite comme étant le quotient d'une mesure de distance et d'un intervalle de temps.
Le cas du mouvement rectiligne uniforme mis à part,
cette définition n'est malheureusement qu'approximative,
bien qu'elle produise des résultats de plus en plus précis
au fur et à mesure que l'intervalle de temps considéré devient petit.
Dans cette optique,
il serait tentant de calculer le quotient sur un intervalle de temps nul,
mais nous sortons malheureusement du domaine de définition du quotient.

Ce problème se manifeste pour de nombreuses quantités physiques.
Pour les traiter toutes simultanément,
nous identifions le problème abstrait sous-jacent suivant:
peut-on généraliser la notion d'image d'un point en dehors du domaine?
C'est pour répondre à cette question fondamentale que le concept de \emph{limite} est introduit.

Les applications des limites sont si vastes
qu'elles ont profondément marqué les mathématiques.
Ainsi, les nombres réels ont été définis pour assurer l'existence des limites.
Par ailleurs,
une volonté d'obtenir des limites plus esthétiques a conduit à l'abandon du degré comme mesure d'angle privilégiée au profit du radian.

Plus simplement,
la notion de limite est fondamentale
car elle constitue le pilier sur lequel repose le \emph{calcul différentiel et intégral}.

\section{Limites en un réel}

\subsection{Adhérence}

Un point \emph{adhérent} à un ensemble est un \emph{réel} qui appartient ou ``colle'' à cet ensemble.
L'ensemble des points adhérents à un ensemble $A \subset \R$
s'appelle l'\emph{adhérence de $A$} et se dénote $\cl A$.

Dans le cadre d'un cours du secondaire,
l'adhérence se calcule facilement:
\begin{howto}
    [Calcul d'adhérence]

    \begin{enumerate}
        \item On réécrit l'ensemble comme union d'intervalles.
        \item L'adhérence contient en plus les bornes \emph{réelles} de chacun des intervalles.
    \end{enumerate}
\end{howto}

En pratique,
nous calculons l'adhérence du domaine d'une fonction.

\begin{example}
    [Adhérence d'un domaine]

    Calculons l'adhérence du domaine de la fonction définie par
    \begin{align}
        f(x) = \sqrt{\frac {x + 3}{x - 10}}.
    \end{align}
    On vérifie que
    \begin{align}
        \dom f = \ocinterval {-\infty} {-3} \cup \oointerval {10} {+\infty},
    \end{align}
    et donc on en conclut que
    \begin{align}
        \cl {\dom f} = \ocinterval {-\infty} {-3} \cup \cointerval {10} {+\infty}.
    \end{align}
\end{example}

Du point de vue du calcul des limites,
les réels les plus intéressants sont les \textbf{points adhérents hors du domaine}.

\subsection{Limite réelle en un réel}

\begin{definition}
    [Limite réelle en un réel]

    \subsubsection*{Cadre}
    \begin{enumerate}
        \item $f$ une fonction;
        \item $a \in \cl {\dom f}$;
        \item $L \in \R$.
    \end{enumerate}

    \subsubsection*{Définition}
    On écrit
    \begin{align}
        \lim_{x \to a} f(x) = L
    \end{align}
    si \textbf{$f(x)$ est aussi proche que l'on veut de $L$}
    lorsque \textbf{$x$ est suffisament proche de $a$ avec $x \in \dom f$}.
\end{definition}

\begin{example}
    [Limite en un réel]

    Supposons que nous considérons la fonction $f$ dont le graphe est esquissé ci-dessous.

    \begin{center}
        \begin{plot}{0.5}{-6}{-6}{6}{6}
            \plotfunction{-6:6}{0.25*\x^3 - 0.8*\x^2 - \x - 1}
            \drawpoint{0, -1}
        \end{plot}
    \end{center}

    Nous remarquons que les images deviennent aussi proches que l'on veut de $-1$
    lorsque les abscisses sont suffisament proches de $0$.
    Nous pouvons écrire
    \begin{align}
        \lim_{x \to 0} f(x) = -1.
    \end{align}
\end{example}

Il faut faire très attention
lorsque une limite est calculée à l'intérieur du domaine.
Nous rompons consciemment avec la plupart des ouvrages mathématiques
en disant que la limite n'existe pas dans l'exemple suivant.

\begin{example}
    [Limite pointée]

    Supposons que nous considérons la fonction $f$ dont le graphe est esquissé ci-dessous.

    \begin{center}
        \begin{plot}{0.5}{-6}{-6}{6}{6}
            \plotfunction{-6:6}{0.25*\x^3 - 0.8*\x^2 - \x - 1}
            \drawpoint{0, -1}
            \drawpoint[black]{0, 2}
        \end{plot}
    \end{center}

    Nous remarquons que les images ne sont pas systématiquement proches de $-1$
    lorsque l'on considère des $x$ proches de $0$.
    En particulier, $x = 0$ est proche de $0$
    alors que son image vaut $2$ qui est loin de $-1$.
    Dès lors,
    la limite $\lim_{x \to 0} f(x)$ n'existe pas.
\end{example}

Nous sentons déjà que je vais me faire traiter d'incompétent par vos profs particuliers,
donc s'il vous plaît défendez-nous et montrez-leur ceci:

\begin{remark}
    [Limite épointée]

    Notre définition coïncide avec la définition classique de limite en dehors du domaine,
    mais peut être différente à l'intérieur.
    À quoi ressemble un cours où la définition de limite inclus le contre-exemple ci-dessus?
    Pour être mathématiquement correct
    (lire: ce que l'Actimath devrait faire),
    un tel cours doit vérifier les points suivants.
    \begin{enumerate}
        \item Toutes les propriétés et définitions doivent être énoncées
            en un réel adhérent et \textbf{non isolé} du domaine.
            Cette hypothèse supplémentaire rallonge littéralement tous les énoncés.
        \item La définition de limite doit être rallongée en excluant explicitement $x = a$.
        \item La définition de \emph{continuité} est plus longue.
        \item La propriété de \emph{composition des limites} serait fausse dans un tel cours sans hypothèse de continuité.
        \item Une personne normale ne calcule que les limites à l'extérieur du domaine, donc on s'en fout.
    \end{enumerate}
\end{remark}

\subsection{Limite infinie en un réel}

\begin{definition}
    [Limite infinie en un réel]

    \subsubsection*{Cadre}
    \begin{enumerate}
        \item $f$ une fonction;
        \item $a \in \cl {\dom f}$;
    \end{enumerate}

    \subsubsection*{Définition}
    On écrit
    \begin{align}
        \lim_{x \to a} f(x) = +\infty \quad
        \left(\text{resp.} \lim_{x \to a} f(x) = -\infty\right)
    \end{align}
    si \textbf{$f(x)$ devient supérieur (resp.\ inférieur) à n'importe quel nombre réel}
    lorsque \textbf{$x$ est suffisament proche de $a$ avec $x \in \dom f$}.
\end{definition}

\begin{example}
    [Limite infinie en un réel]

    Supposons que nous considérons la fonction $f$ dont le graphe est esquissé ci-dessous.

    \begin{center}
        \begin{plot}{0.5}{-6}{-1}{6}{12}
            \plotfunction{-6:6}{1/(\x - 1)^2}
        \end{plot}
    \end{center}

    Nous remarquons que les images deviennent aussi grande que l'on souhaite
    lorsque les abscisses sont suffisament proches de $1$.
    Nous pouvons écrire
    \begin{align}
        \lim_{x \to 1} f(x) = +\infty.
    \end{align}
\end{example}

\subsection{Limites latérales}

\begin{example}
    [Fonctions homographiques]

    Il est clair que la fonction $f$ définie par
    \begin{align}
        f(x) = \frac 1 {x + 1},
    \end{align}
    dont le graphe est ci-dessous,
    \begin{center}
        \begin{plot}{0.5}{-6}{-6}{6}{6}
            \plotfunction{-6:0.99}{1/(\x - 1)}
            \plotfunction{1.01:6}{1/(\x - 1)}
        \end{plot}
    \end{center}
    n'a pas de limite, même infinie, en $1$.

    Nous serions tenté de dire que la limite en $1$ par la gauche est $-\infty$,
    tandis que la limite en $1$ par la droite est $+\infty$.
    Nous noterons ça:
    \begin{align}
        \lim_{x \to 1^-} f(x) = -\infty
        \quad
        \lim_{x \to 1^+} f(x) = +\infty
    \end{align}
\end{example}

\subsection{Interprétation graphique}

Trois cas se présentent lorsqu'une limite existe en un réel.

\begin{enumerate}
    \item Si la limite existe en $a \in \dom f$,
        alors on dit que $f$ est \textbf{continue} en $a$.
        Par exemple,
        dans le graphe ci-dessous,
        \begin{center}
            \begin{plot}{0.5}{-6}{-1}{6}{3}
                \plotfunction{-6:6}{0.05*\x^3 + 1}
                \drawpoint[black]{0, 1}
            \end{plot}
        \end{center}
        on a $\lim_{x \to 0} f(x) = 1$ donc $f$ est \emph{continue} en $0$.
    \item Si la limite existe en $a \notin \dom f$ et est réelle,
        alors $(a, \lim_{x \to a} f(x))$ est un \textbf{point creux}.
        Par exemple,
        dans le graphe ci-dessous,
        \begin{center}
            \begin{plot}{0.5}{-6}{-2}{6}{2}
                \plotfunction{-6:6}{0.05*\x^3}
                \drawpoint{0, 0}
            \end{plot}
        \end{center}
        on a $\lim_{x \to 0} f(x) = 0$ donc $(0, 0)$ est un \emph{point creux}.
    \item Si une limite latérale en $a$ est infinie,
        alors nous avons une \textbf{asymptote verticale} en $a$.
        Intuitivement,
        on voit que proche de $a$,
        le graphe de $f$ se rapproche de la droite d'équation
        \begin{align}
            AV_a \equiv x = a.
        \end{align}
        Dans l'exemple suivant,
        \begin{center}
            \begin{plot}{0.25}{-6}{-1}{6}{12}
                \plotfunction{-6:6}{1/(\x - 1)^2 + 2}
                \drawline (1, -1) -- (1, 13);
            \end{plot}
        \end{center}
        on remarque que $\lim_{x \to 1} f(x) = +\infty$,
        et qu'autour de $1$,
        le graphe est proche de l'\emph{asymptote verticale} d'équation
        \begin{align}
            AV_1 \equiv x = 1.
        \end{align}
\end{enumerate}

\begin{definition}
    [Asymptote verticale]

    Soit $f$ une fonction.
    Si une limite latérale de $f$ en $a$ a un sens, existe et est infinie,
    alors la droite d'équation
    \begin{align}
        AV_a \equiv x = a
    \end{align}
    est une \emph{asymptote verticale} de $f$.
\end{definition}

\subsection{Propriétés des limites}

Les limites se comportent bien avec les opérations algébrique.

\begin{howto}
    [Opérations sur les limites]

    \begin{itemize}
        \item On a:
            \begin{align}
                \lim_{x \to a} (f + g)(x) = \lim_{x \to a} f(x) + \lim_{x \to a} g(x)
            \end{align}
            si les limites apparaîssant à droite \emph{existent et sont réelles},
            et que la limite à gauche a un sens.
        \item On a:
            \begin{align}
                \lim_{x \to a} (f - g)(x) = \lim_{x \to a} f(x) - \lim_{x \to a} g(x)
            \end{align}
            si les limites apparaîssant à droite \emph{existent et sont réelles},
            et que la limite à gauche a un sens.
        \item On a:
            \begin{align}
                \lim_{x \to a} (f g)(x) = \left(\lim_{x \to a} f(x)\right) \left(\lim_{x \to a} g(x)\right)
            \end{align}
            si les limites apparaîssant à droite \emph{existent et sont réelles},
            et que la limite à gauche a un sens.
        \item On a:
            \begin{align}
                \lim_{x \to a} \frac f g(x) = \frac {\lim_{x \to a} f(x)} {\lim_{x \to a} g(x)}
            \end{align}
            si les limites apparaîssant à droite \emph{existent et sont réelles},
            que la limite à gauche a un sens,
            et la limite du dénominateur est non nulle.
        \item Si $g$ est une fonction élémentaire (au sens du chapitre précédent),
            alors
            \begin{align*}
                \lim_{x \to a} g(f(x)) = g(\lim_{x \to a} f(x)),
            \end{align*}
            à condition que le membre de droite soit bien défini,
            et que la limite à gauche a un sens.
    \end{itemize}
\end{howto}

\begin{remark}
    Nous insistons sur le fait que l'hypothèse ci-dessus
    que les limites existent et soient réelles
    est cruciale.

    Ceci est du au fait
    que les règles arithmétiques usuelles ne s'appliquent pas à $\pm \infty$.
    Par exemple,
    nous avons
    \begin{align}
        0 &= \lim_{x \to 0} \frac {1 - 1} {x^2} \neq \lim_{x \to 0} \frac 1 {x^2} - \lim_{x \to 0} \frac 1 {x^2}\\
        +\infty &= \lim_{x \to 0} \frac {2 - 1} {x^2} \neq \lim_{x \to 0} \frac 2 {x^2} - \lim_{x \to 0} \frac 1 {x^2}\\
        -\infty &= \lim_{x \to 0} \frac {1 - 2} {x^2} \neq \lim_{x \to 0} \frac 1 {x^2} - \lim_{x \to 0} \frac 2 {x^2}
    \end{align}
    alors que les membres de droite valent tous formellement $+\infty - \infty$.
\end{remark}

\subsection{Calcul de limites en un réel}

Dans toute cette section,
nous supposerons que nous devons calculer la limite
\begin{align}
    \lim_{x \to a} \frac {n(x)} {d(x)},
\end{align}
où les lettres $n$ et $d$ ont été choisies
pour signifier le \emph{numérateur} et \emph{dénominateur} respectivement.

\begin{enumerate}
    \item Si le dénominateur ne s'annule pas en $a$,
        alors la limite s'obtient en évaluant la fraction en $a$.
        \begin{align}
            \lim_{x \to a} \frac {n(x)} {d(x)} = \frac {n(a)} {d(a)}.
        \end{align}
    \item Si seul le dénominateur s'annule en $a$
        (on parle alors d'indétermination de type $\left[\frac k 0\right]$),
        alors les limites latérales en $a$ sont \emph{infinies}
        et on utilise un tableau de signe pour déterminer les signes de ces limites
        (voir ci-dessous).
    \item Si le numérateur et le dénominateur s'annulent en $a$,
        alors on applique les étapes suivantes.
        \begin{enumerate}
            \item On se débarasse des racines carrées gênantes via la technique du binôme conjugué;
            \item On met en évidence le facteur $(x - a)$ dans $n$ et $d$ pour ensuite le simplifier.
            \item On recommence.
        \end{enumerate}
\end{enumerate}

\subsubsection{Indétermination $k$ sur $0$}

Nous sommes dans le cas où seul le dénominateur s'annule en $a$.
Pour $x$ proche de $a$ mais distinct de $a$,
$d(x)$ sera très proche de $0$ mais non nul,
rendant le quotient $\frac {n(x)} {d(x)}$ très grand (en valeur absolue).
Il suit alors que les limites
\begin{align}
    \lim_{x \to a^-} \frac {n(x)} {d(x)},
    \quad
    \lim_{x \to a^+} \frac {n(x)} {d(x)}
\end{align}
seront infinies,
et qu'il ne reste qu'à déterminer le signe des limites ci-dessus.
Pour ce faire,
nous utiliserons un tableau de signe.

Remarquons que le numérateur $n$ ne change pas de signe autour de $a$ (puisque $n(a) \neq 0$).
Dès lors,
les expressions
\begin{align}
    \frac {n(x)} {d(x)}
    \quad \text{et} \quad
    \frac {n(a)} {d(x)}
\end{align}
ont le même signe proche de $a$,
et pour des raisons pratiques nous étudierons le signe de la deuxième.

\begin{example}
    [Indétermination $k$ sur $0$]

    Supposons que nous voulions calculer la limite
    \begin{align}
        \lim_{x \to 0} \frac 1 x.
    \end{align}

    En évaluant en $x = 0$,
    nous obtenons formellement $\left[\frac {1} {0}\right]$.
    Nous savons dès lors que les limites latérales en $1$ sont infinies.

    Le tableau de signe
    \signtable{$x$/0.75, $1$/0.75, $x$/0.75, $\frac 1 x$/0.75}{,$0$,}{%
        \signrow{,+,+,+,}
        \signrow{,-,0,+,}
        \signrow{,-,d,+,}
    }
    nous permet de déduire (en regardant les signes aux alentours de $1$) que
    \begin{align}
        \lim_{x \to 0^-} \frac 1 x &= -\infty\\
        \lim_{x \to 0^+} \frac 1 x &= +\infty.
    \end{align}

    Cela peut se vérifier graphiquement:
    \begin{center}
        \begin{plot}{0.25}{-12}{-12}{12}{12}
            \plotfunction{-12:-0.01}{1/\x}
            \plotfunction{0.01:12}{1/\x}
        \end{plot}
    \end{center}
\end{example}

\begin{example}
    [Indétermination $k$ sur $0$]

    Supposons que nous voulions calculer la limite
    \begin{align}
        \lim_{x \to 1} \frac {x^3 - 7x - 6} {x^2 - 3x + 2}.
    \end{align}

    En évaluant en $x = 1$,
    nous obtenons formellement $\left[\frac {-12} {0}\right]$.
    Nous savons dès lors que les limites latérales en $1$ sont infinies.

    Le tableau de signe
    \signtable{$x$/0.75, $-12$/0.75, $x^2 - 3x + 2$/0.75, $\frac {-12} {x^2 - 3x + 2}$/0.75}{,$1$, $2$,}{%
        \signrow{,-,-,-,-,-,}
        \signrow{,+,0,-,0,+,}
        \signrow{,-,d,+,d,-,}
    }
    nous permet de déduire (en regardant les signes aux alentours de $1$) que
    \begin{align}
        \lim_{x \to 1^-} \frac {x^3 - 7x - 6} {x^2 - 3x + 2} &= -\infty,\\
        \lim_{x \to 1^+} \frac {x^3 - 7x - 6} {x^2 - 3x + 2} &= +\infty.
    \end{align}
\end{example}

\subsubsection{Factorisation}

Si le numérateur et le dénominateur sont des polynômes s'annulant tous les deux en $a$,
alors $x - a$ est un facteur commun du numérateur et du dénominateur.
On peut alors le simplifier.

\begin{example}
    [Indétermination $0$ sur $0$]

    Calculons la limite
    \begin{align}
        \lim_{x \to 1} \frac {2x^2 + x - 3} {3x^2 - 8x + 5}.
    \end{align}

    On vérifie qu'une évaluation en $1$ donne $[\frac 0 0]$.
    En factorisant, on obtient
    \begin{align}
        \lim_{x \to 1} \frac {2x^2 + x - 3} {3x^2 - 8x + 5}
        &= \lim_{x \to 1} \frac {(x - 1)(2x + 3)} {(x - 1)(3x - 5)}\\
        &= \lim_{x \to 1} \frac {2x + 3} {3x - 5} = - \frac 5 2.
    \end{align}
\end{example}

Remarquons qu'il est aussi possible d'obtenir des cas moins favorables après la simplification.

\begin{example}
    [Indétermination $0$ sur $0$]

    Calculons la limite
    \begin{align}
        \lim_{x \to 1} \frac {2x^2 + x - 3} {x^2 - 2x + 1}.
    \end{align}

    On vérifie qu'une évaluation en $1$ donne $[\frac 0 0]$.
    En factorisant, on obtient
    \begin{align}
        \lim_{x \to 1} \frac {2x^2 + x - 3} {x^2 - 2x + 1}
        &= \lim_{x \to 1} \frac {(x - 1)(2x + 3)} {{(x - 1)}^2}\\
        &= \lim_{x \to 1} \frac {2x + 3} {x - 1} = \left[\frac 5 0\right].
    \end{align}

    En faisant le tableau de signe,
    \signtable{$x$/0.75, $5$/0.75, $x - 1$/0.75, $\frac 5 {x - 1}$/0.75}{,$1$,}{%
        \signrow{,+,+,+,}
        \signrow{,-,0,+,}
        \signrow{,-,d,+,}
    }
    nous en déduisons que
    \begin{align}
        \lim_{x \to 1^-} \frac {2x^2 + x - 3} {x^2 - 2x + 1} &= -\infty\\
        \lim_{x \to 1^+} \frac {2x^2 + x - 3} {x^2 - 2x + 1} &= +\infty
    \end{align}
\end{example}

\subsubsection{Technique du binôme conjugué}

Si le numérateur et le dénominateur s'annulent tous les deux en $a$,
et que l'un deux comporte des racines,
alors on multiplie le numérateur et le dénominateur par le binôme conjugué du terme avec les racines.
L'indétermination $0$ sur $0$ persistera,
mais elle ne sera plus causée par des racines.

\begin{example}
    [Indétermination $0$ sur $0$ avec des racines carrées]

    Calculons la limite
    \begin{align}
        \lim_{x \to 0} \frac {\sqrt{x^2 + x + 2} - \sqrt{2}} {x^2 - x}.
    \end{align}

    Une évaluation en $0$ donne bien $[\frac 0 0]$,
    et on constate qu'une racine carrée apparaît au numérateur.
    Nous multiplions dès lors numérateur et dénominateur par le binôme conjugué
    \begin{align}
        \sqrt {x^2 + x + 2} + \sqrt 2,
    \end{align}
    ce qui donne
    \begin{align}
        \lim_{x \to 0} \frac {\sqrt{x^2 + x + 2} - \sqrt{2}} {x^2 - x}
        &= \lim_{x \to 0} \frac {\sqrt{x^2 + x + 2} - \sqrt{2}} {x^2 - x} \frac {\sqrt {x^2 + x + 2} + \sqrt 2} {\sqrt {x^2 + x + 2} + \sqrt 2}\\
        &= \lim_{x \to 0} \frac {x^2 + x} {(x^2 - x)(\sqrt {x^2 + x + 2} + \sqrt 2)}.
    \end{align}

    Remarquons qu'on a une indétermination $0$ sur $0$
    mais cette dernière est causée maintenant par un quotient de polynôme.
    En factorisant, on obtient
    \begin{align}
        \lim_{x \to 0} \frac {\sqrt{x^2 + x + 2} - \sqrt{2}} {x^2 - x}
        &= \lim_{x \to 0} \frac {x(x + 1)} {x(x - 1)(\sqrt {x^2 + x + 2} + \sqrt 2)}\\
        &= \lim_{x \to 0} \frac {(x + 1)} {(x - 1)(\sqrt {x^2 + x + 2} + \sqrt 2)}\\
        &= -\frac 1 {2 \sqrt 2}.
    \end{align}
\end{example}

\section{Limites à l'infini}

\begin{definition}
    [Limite réelle à l'infini]

    Soit $f$ une fonction.
    Soit $L \in \R$ et supposons que $\dom f$ ne soit pas majoré.
    On écrit
    \begin{align}
        \lim_{x \to +\infty} f(x) = L
    \end{align}
    si \emph{$f(x)$ est aussi proche que l'on veut de $L$}
    lorsque \emph{$x$ est suffisament grand avec $x \in \dom f$}.
\end{definition}

\begin{example}
    [Limite réelle à l'infini]

    Considérons la fonction $f$ dont le graphe est esquissé ci-dessous.
    \begin{center}
        \begin{plot}{0.25}{-1}{-3}{30}{5}
            \plotfunction{-1:30}{10*sin(90*\x)/(\x + 5) + 1}
        \end{plot}
    \end{center}

    Nous voyons que les images associées à des $x$ grands ont tendance à rester de plus en plus proche de $1$.
    Nous pouvons alors écrire
    \begin{align*}
        \lim_{x \to +\infty} f(x) = 1.
    \end{align*}

    Nous voyons également que le graphe reste de plus en plus près de la droite d'équation $y = 1$,
    comme le montre la figure suivante:
    \begin{center}
        \begin{plot}{0.25}{-1}{-3}{30}{5}
            \plotfunction{-1:30}{10*sin(90*\x)/(\x + 5) + 1}
            \plotfunction{-1:30}{1}
        \end{plot}
    \end{center}

    Cette droite s'appelle une \emph{asymptote horizontale}.
\end{example}

\begin{definition}
    [Asymptote horizontale]

    Soit $f$ une fonction.
    Si la limite à l'infini (resp.\ moins l'infini) de $f$ a un sens, existe et vaut $L \in \R$,
    alors la droite d'équation
    \begin{align}
        AV_{+\infty} \equiv y = L
        \quad
        \left(\text{resp.}\ AV_{-\infty} \equiv y = L\right)
    \end{align}
    est une \emph{asymptote horizontale} de $f$.
\end{definition}

\begin{definition}
    [Asymptote oblique]

    Soit $f$ une fonction et $m, p \in \R$.
    Si la limite
    \begin{align*}
        \lim_{x \to +\infty} \left(f(x) - [mx + p]\right)
        \quad \left(\text{resp.}\ \lim_{x \to -\infty} \left(f(x) - [mx + p]\right)\right)
    \end{align*}
    a un sens, existe et vaut $0$,
    alors la droite d'équation
    \begin{align}
        AO_{+\infty} \equiv y = mx + p
        \quad \left(\text{resp.}\ AO_{-\infty} \equiv y = mx + p\right)
    \end{align}
    est une \emph{asymptote oblique} de $f$.
\end{definition}

\begin{example}
    [Asymptote oblique]

    Considérons la fonction $f$ définie par
    \begin{align}
        f(x) = \frac 1 2 x + 1 + \frac 1 x.
    \end{align}

    Nous voyons que
    \begin{align}
        \lim_{x \to \pm \infty} (f(x) - (\frac 1 2 x + 1)) = \lim_{x \to \pm \infty} \frac 1 x = 0,
    \end{align}
    ce qui signifie que la droite
    \begin{align}
        AO \equiv y = \frac 1 2 x + 1
    \end{align}
    est une \emph{asymptote oblique}.

    Nous vérifions graphiquement la tendance du graphe à s'éloigner de moins en moins de l'asymptote pour les grandes valeurs.
    \begin{center}
        \begin{plot}{0.25}{-12}{-12}{12}{12}
            \plotfunction{-12:12}{0.5*\x + 1 + 1/\x}
            \plotfunction{-12:12}{0.5*\x + 1}
        \end{plot}
    \end{center}
\end{example}

\subsection{Calcul des limites à l'infini}

\subsubsection{Limite de $k x^n$, $k \neq 0$}

Les limites à l'infini de $k x^n$, $k \neq 0$ sont infinies.
Le signe se détermine par une étude de signe pour les très grandes (resp.\ petites) valeurs.
De manière très (trop\footnote{Si j'expliquais bien, je n'encouragerais pas ça.}) pragmatique,
on a
\begin{align*}
    \lim_{x \to +\infty} k x^n &= (\sign k) \infty\\
    \lim_{x \to -\infty} k x^n &=
    \begin{cases}
        (-\sign k) \infty & \text{si}\ n\ \text{est impair,} \\
        (\sign k) \infty & \text{sinon.}
    \end{cases}
\end{align*}

\subsubsection{Limite de $\frac k {x^n}$, $k \neq 0$}

Ca vaut toujours $0$,
vu qu'on divise $k$ par quelque chose qui devient de plus en plus grand.
\begin{align}
    \lim_{x \to \pm \infty} \frac k {x^n} = 0.
\end{align}

\subsubsection{Règle de la plus haute puissance}

\begin{remark}
    [Règle de la plus haute puissance]

    Dans le calcul des limites à l'infini,
    seul le \emph{terme avec la plus haute puissance} compte.

    On se débarasse progressivement des termes avec des puissances plus basses,
    dans l'ordre de la priorité des opérations (sans tenir compte des parenthèses)
    (d'abord dans les puissances et racines, puis dans les numérateurs/dénominateurs, etc.).
\end{remark}

\begin{example}
    [Limite à l'infini]

    Calculons la limite
    \begin{align*}
        \lim_{x \to \pm \infty} \frac {3x^2 + 4 x^9 - 2 x^7 + 4} {-2x^3 + 2x^5 + 3}.
    \end{align*}

    En appliquant la règle de la plus haute puissance au numérateur et au dénominateur,
    on obtient
    \begin{align*}
        \lim_{x \to \pm \infty} \frac {3x^2 + 4 x^9 - 2 x^7 + 4} {-2x^3 + 2x^5 + 3}
        &= \lim_{x \to \pm \infty} \frac {4 x^9} {2 x^5}\\
        &= \lim_{x \to \pm \infty} 2 x^4 = +\infty.
    \end{align*}
\end{example}

\begin{example}
    [Limite à l'infini]

    Calculons la limite
    \begin{align}
        \lim_{x \to \pm \infty} \frac {\sqrt{4x^4 + 5x} + x^2 - 5} {3 x^2 - 5x + 2}.
    \end{align}

    En appliquant la règle de la plus haute puissance à la racine,
    puis au numérateur et au dénominateur,
    on obtient
    \begin{align}
        \lim_{x \to \pm \infty} \frac {\sqrt{4x^4 + 5x} + x^2 - 5} {3 x^2 - 5x + 2}
        &= \lim_{x \to \pm \infty} \frac {\sqrt{4x^4} + x^2 - 5} {3 x^2 - 5x + 2}\\
        &= \lim_{x \to \pm \infty} \frac {3 x^2 - 5} {3 x^2 - 5x + 2} = 1.
    \end{align}
\end{example}

\subsubsection{Asymptotes obliques}

Dans le cadre du cours de mathématiques $4$ heures,
le cas ne se présente que lorsque nous traitons un quotient de type
\begin{align}
    \lim_{x \to \pm \infty} \frac {P_n(x)} {P_{n - 1}(x)},
\end{align}
où $P_n$ et $P_{n - 1}$ désignent des polynômes de degrés $n$ et $n - 1$ respectivement.

Pour trouver l'équation de l'asymptote oblique,
on effectue alors la division euclidienne de $P_n$ par $P_{n - 1}$.
On obtient alors
\begin{align}
    \frac {P_n(x)} {P_{n - 1}(x)} = Q_1(x) + \frac {R_{n - 2}(x)} {P_{n - 1}(x)},
\end{align}
où $Q_1$ et $R_{n - 2}$ sont des polynômes de degré $1$ et $n - 2$ respectivement.

L'équation de l'asymptote oblique est dès lors donnée par
\begin{align}
    AO \equiv y = Q_1(x),
\end{align}
puisque l'on vérifie aisément que
\begin{align}
    \lim_{x \to \pm \infty} \left(\frac {P_n(x)} {P_{n - 1}(x)} - Q_1(x)\right)
    = \lim_{x \to \pm \infty} \frac {R_{n - 2}(x)} {P_{n - 1}(x)} = 0.
\end{align}

\begin{example}
    [Asymptote oblique]

    Supposons que nous devons déterminer l'asymptote oblique du quotient
    \begin{align}
        f(x) = \frac {3x^2 + 5x + 7} {2 x + 4}.
    \end{align}

    On vérifie tout d'abord que la différence des degrés entre le numérateur et le dénominateur est bien de $1$.
    Ensuite,
    nous effectuons la division euclidienne pour obtenir:
    \begin{align}
        f(x) = \frac 3 2 x - \frac 1 2 + \frac 9 {2 x + 4}
    \end{align}
    de telle sorte que l'asymptote oblique est bien
    \begin{align}
        AO \equiv \frac 3 2 x - \frac 1 2.
    \end{align}
\end{example}

\section{Résumé}

En combinant les lignes du tableau~\ref{table:limit_definition},
on peut déduire chacune des définitions de limites.

\begin{definition}
    [Limite]

    Soit $f$ une fonction.
    \textbf{[Hypothèses supplémentaires]}.
    On écrit \textbf{[limite à définir]}
    si \textbf{[comportement des ordonnées]}
    lorsque \textbf{[comportement des abscisses]}.
\end{definition}

\begin{sidewaystable}
    \centering
    \caption{Tableau récapitulatif pour les définitions de limite}\label{table:limit_definition}
    \begin{tabular}
        {l l l l}
        \toprule
        Maths & se lit\dots & Hypothèses & En français \\ \midrule
        $\lim f(x) = L$ & $f(x)$ tend vers $L$ & $L \in \R$ & $f(x)$ est aussi proche que l'on veut de $L$ \\
        $\lim f(x) = +\infty$ & $f(x)$ tend vers $+\infty$ & & $f(x)$ devient supérieur à n'importe quel réel\\
        $\lim f(x) = -\infty$ & $f(x)$ tend vers $-\infty$ & & $f(x)$ devient inférieur à n'importe quel réel\\
        $x \to a$ & $x$ tend vers $a$ & $a \in \cl {\dom f}$ & $x$ est suffisament proche de $a$, avec $x \in \dom f$\\
        $x \to a^+$ & $x$ tend vers $a$ par la droite & $a \in \cl {\dom f \cap \cointerval{a}{+\infty}}$ & $x$ est suffisament proche de $a$ avec $x \in \dom f$ et $x \geq a$\\
        $x \to a^-$ & $x$ tend vers $a$ par la gauche & $a \in \cl {\dom f \cap \ocinterval{-\infty}{a}}$ & $x$ est suffisament proche de $a$ avec $x \in \dom f$ et $x \leq a$\\
        $x \to +\infty$ & $x$ tend vers $+\infty$ & $\dom f$ non majoré & $x$ est suffisament grand avec $x \in \dom f$\\
        $x \to -\infty$ & $x$ tend vers $-\infty$ & $\dom f$ non minoré & $x$ est suffisament petit avec $x \in \dom f$\\
        \bottomrule
    \end{tabular}
\end{sidewaystable}

\subsection{Exercice type}

L'exercice typique est de vous donner une fonction,
en vous demandant de calculer toutes les limites importantes
et de donner les interprétations graphiques.

Les limites importantes sont:
\begin{itemize}
    \item $-\infty$, $+\infty$;
    \item tous les points adhérents \emph{hors} du domaine.
\end{itemize}

\subsubsection{Calcul de limite en un réel}

\tikzset{%
    bordered/.style = {shape=rectangle, draw}
}
\begin{tikzpicture}
    [
        grow                    = right,
        sibling distance        = 5em,
        level distance          = 10em,
        edge from parent/.style = {draw, -latex},
        sloped
    ]
    \node (E) {Évaluation en $a$}
    child {%
        node {$\left[\dfrac 0 0\right]$}
        child {%
            node (F) {factorisation}
            child {%
                node (S) {simplification}
            }
        }
        child {%
            node (B) {binôme conjugué}
            edge from parent node [above] {$\sqrt{\ }$?}
        }
    }
    child {%
        node {$\left[\dfrac k 0\right]$, $k \neq 0$}
        child {%
            node {TS de $\dfrac k {d(x)}$}
            child {%
                node [bordered] {$AV_a \equiv x = a$}
            }
        }
    }
    child {%
        node {$L \in \R$}
        child {%
            node [bordered] {Point creux: $(a, L)$}
            edge from parent node [above] {$a \notin \dom$}
        }
        child {%
            node [bordered] {continue en $a$}
            edge from parent node [above] {$a \in \dom$}
        }
    };
    \draw [->,-latex] (B) to (F);
    \draw [->,-latex] (S) to [out=225, in=270] (E);
\end{tikzpicture}

\subsubsection{Calcul de limite à l'infini}

Dans le schéma ci-dessous,
PHP fait référence à la règle de la \emph{plus haute puissance}.

Si on est dans le cas
\begin{align}
    \lim_{x \to \pm \infty} \frac {P_n(x)} {P_{n - 1}(x)},
\end{align}
où $P_n$ et $P_{n - 1}$ sont des polynômes de degrés $n$ et $n - 1$ respectivement,
alors il faut calculer l'\emph{asymptote oblique}.

\begin{tikzpicture}
    [
        grow                    = right,
        sibling distance        = 5em,
        level distance          = 10em,
        edge from parent/.style = {draw, -latex},
        sloped
    ]
    \node (P) {PHP}
    child {%
        node (S) {Simplifications}
        child {%
            node {$\begin{aligned}\lim_{x \to \pm \infty} k = k\end{aligned}$}
            child {%
                node [bordered] {$AH \equiv y = k$}
            }
        }
        child {%
            node {$\begin{aligned}\lim_{x \to \pm \infty} \frac k {x^n} = 0\end{aligned}$}
            child {%
                node [bordered] {$AH \equiv y = 0$}
            }
        }
        child {%
            node {$\begin{aligned}\lim_{x \to \pm \infty} k x^n\end{aligned}$, $k \neq 0$}
            child {%
                node {$(\sign k) \infty$}
                edge from parent node [below] {sinon}
            }
            child {%
                node {$(-\sign k) \infty$}
                edge from parent node [above] {$n$ impair}
                edge from parent node [below] {$-\infty$}
            }
        }
    };
    \draw [->,-latex] (S) to [out=270, in=270] (P);
\end{tikzpicture}

\end{document}
