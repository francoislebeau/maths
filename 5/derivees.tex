\documentclass[main.tex]{subfiles}
\begin{document}

\chapter{Dérivées}

\section{Introduction}

L'introduction du concept de dérivée est historiquement importante,
puisqu'elle coïncide avec la naissance de l'analyse mathématique et de la mécanique classique de Newton.
Cette année-là\footnote{%
    L'année 1666 est souvent qualifiée de \emph{miraculeuse} (\emph{annus mirabilis} pour les prétentieux)
    car Newton fait des avancées considérables dans l'étude de
    l'analyse mathématique, la dynamique, l'optique et la gravitation.

    Un autre example d'\emph{annus mirabilis} est l'année 2009 du FC Barcelone.
    Ils ont gagné la Liga, la Coupe, la Super Coupe, la Champions League, la Super Coupe d'Europe et la Coupe du monde des clubs.
    Cet exploit leur permet d'être inconstestablement considéré comme le deuxième plus grand club de football d'Espagne.
},
les mathématiques ont acquis le pouvoir d'exprimer les lois naturelles et de modéliser des situations complexes.

\section{Définition et premières propriétés}

\begin{definition}
    [Dérivée]

    \subsubsection*{Cadre}

    \begin{itemize}
        \item $f$ est une fonction;
        \item $a \in \dom f$ est un point \emph{non-isolé} de $\dom f$.
    \end{itemize}

    \subsubsection*{Définition}

    Si la limite
    \begin{align}
        \lim_{x \to a} \frac {f(x) - f(a)} {x - a}
    \end{align}
    existe et est réelle,
    alors le nombre
    \begin{align}
        f'(a) \defeq \lim_{x \to a} \frac {f(x) - f(a)} {x - a}
    \end{align}
    est appelé la \emph{dérivée} de $f$ en $a$.
\end{definition}

\begin{remark}
    [Dérivabilité]

    \begin{itemize}
        \item Nous dirons qu'une fonction $f$ est \textbf{dérivable en $a$}
            si la dérivée de $f$ en $a$ a un sens et existe.
        \item Nous dirons qu'une fonction $f$ est \textbf{dérivable}
            si elle est dérivable en tout point de son domaine.
    \end{itemize}
\end{remark}

\begin{remark}
    Une expression telle que
    \begin{align}
        \frac {f(x) - f(a)} {x - a}
    \end{align}
    est appelée \emph{quotient différentiel}.
    Dans ce langage,
    une dérivée est une limite d'un quotient différentiel.
\end{remark}

\begin{definition}
    [Tangente]

    Soit $f$ une fonction dérivable en un réel $a \in \R$.
    La \emph{tangente} de $f$ en $a$ est la droite de pente $f'(a)$ passant par $(a, f(a))$.
    Autrement dit,
    la tangente est la droite d'équation
    \begin{align}
        T_{f, a} \equiv f'(a) (x - a) + f(a).
    \end{align}
\end{definition}

\begin{example}
    [Dérivée d'une constante]

    Soit $k \in \R$ une constante.
    Calculons la dérivée de $f(x) = k$ en un $a \in \R$ quelconque.
    Dans ce but,
    analysons le quotient différentiel
    \begin{align}
        \frac {f(x) - f(a)}{x - a}
        &= \frac {k-k}{x - a}\\
        &= 0.
    \end{align}

    En faisant tendre $x$ vers $a$ des deux côtés,
    on obtient
    \begin{align}
        f'(a) = 0.
    \end{align}
\end{example}

\begin{example}
    [Dérivée de $x$]

    Calculons la dérivée de $f(x) = x$ en un $a \in \R$ quelconque.
    Dans ce but,
    analysons le quotient différentiel
    \begin{align}
        \frac {f(x) - f(a)}{x - a}
        &= \frac {x - a}{x - a}\\
        &= 1.
    \end{align}

    En faisant tendre $x$ vers $a$ des deux côtés,
    on obtient
    \begin{align}
        f'(a) = 1.
    \end{align}
\end{example}

\begin{example}
    [Dérivée de $x^2$]

    Calculons la dérivée de $f(x) = x^2$ en un $a \in \R$ quelconque.
    Dans ce but,
    analysons le quotient différentiel
    \begin{align}
        \frac {f(x) - f(a)}{x - a}
        &= \frac {x^2 - a^2}{x - a}\\
        &= \frac {(x + a)(x - a)}{x - a}\\
        &= x + a
    \end{align}

    En faisant tendre $x$ vers $a$ des deux côtés,
    on obtient
    \begin{align}
        f'(a) = 2a.
    \end{align}
\end{example}

\begin{example}
    [Dérivée de $x^3$]

    Calculons la dérivée de $f(x) = x^3$ en un $a \in \R$ quelconque.
    Dans ce but,
    analysons le quotient différentiel
    \begin{align}
        \frac {f(x) - f(a)}{x - a}
        &= \frac {x^3 - a^3}{x - a}\\
        &= \frac {(x-a)(x^2 + ax + a^2)}{x - a}\\
        &= x^2 + ax + a^2.
    \end{align}

    En faisant tendre $x$ vers $a$ des deux côtés,
    on obtient
    \begin{align}
        f'(a) = 3a^2.
    \end{align}
\end{example}

On peut extrapoler l'exemple suivant.

\begin{example}
    [Dérivée de $x^n$]

    Soit $n \in \N$.
    Calculons la dérivée de $f(x) = x^n$ en un $a \in \R$ quelconque.
    Dans ce but,
    analysons le quotient différentiel
    \begin{align}
        \frac {f(x) - f(a)}{x - a}
        &= \frac {x^n - a^n}{x - a}\\
        &= \frac {(x-a)(x^{n-1} + ax^{n-2} + a^2x^{n-3} + \cdots + a^{n-2}x + a^{n-1})}{x - a}\\
        &= (x^{n-1} + ax^{n-2} + a^2x^{n-3} + \cdots + a^{n-2}x + a^{n-1}).
    \end{align}

    En faisant tendre $x$ vers $a$ des deux côtés,
    on obtient
    \begin{align}
        f'(a) = na^{n-1}.
    \end{align}
\end{example}

\begin{example}
    [Dérivée de $\sqrt x$]

    Calculons la dérivée de $f(x) = \sqrt x$ en un $a \in \R^+$ quelconque.
    Dans ce but,
    analysons le quotient différentiel
    \begin{align}
        \frac {f(x) - f(a)}{x - a}
        &= \frac {\sqrt x - \sqrt a}{x - a}\\
        &= \frac {\sqrt x - \sqrt a}{x - a} \cdot \frac{\sqrt x + \sqrt a}{\sqrt x + \sqrt a}\\
        &= \frac {{(\sqrt x)}^2 - {(\sqrt a)}^2}{(x - a) (\sqrt x + \sqrt a)}\\
        &= \frac {x - a} {(x - a) (\sqrt x + \sqrt a)}\\
        &= \frac 1 {\sqrt x + \sqrt a}
    \end{align}

    En faisant tendre $x$ vers $a$ des deux côtés,
    on obtient
    \begin{align}
        f'(a) = \frac 1 {2 \sqrt a}
    \end{align}
\end{example}

\begin{example}
    [Dérivée de $\frac 1 x$]

    Calculons la dérivée de $f(x) = \frac 1 x$ en un $a \in \R_0$ quelconque.
    Dans ce but,
    analysons le quotient différentiel
    \begin{align}
        \frac {f(x) - f(a)} h
        &= \frac {\frac 1 x - \frac 1 a}{x - a}\\
        &= \frac{\frac{a - x}{ax}}{x - a}\\
        &= \frac {-1}{ax}
    \end{align}

    En faisant tendre $x$ vers $a$ des deux côtés,
    on obtient
    \begin{align}
        f'(a) = \frac {-1} {a^2}.
    \end{align}
\end{example}

\begin{proposition}
    [Dérivabilité et continuité]

    Soit $f$ une fonction dérivable en $a \in \R$.
    La limite
    \begin{align}
        \lim_{x \to a} f(x)
    \end{align}
    existe et vaut $f(a)$
    (autrement dit, $f$ est \emph{continue} en $a$).
\end{proposition}
\begin{proof}
    Soit $x \in \dom f \setminus \{a\}$.
    On vérifie que
    \begin{align}
        f(x) = \left(\frac {f(x) - f(a)} {x - a}\right) (x - a) + f(a).
    \end{align}

    En faisant tendre $x$ vers $a$ des deux côtés\footnote{%
        Techniquement, on fait tendre $x \to a$ avec la contrainte supplémentaire $x \neq a$.
        En d'autres termes, on calcule une limite \emph{épointée}.
        Il faut ensuite conclure qu'au vu du résultat obtenu (i.e.\ $f(a)$),
        la contrainte peut être levée.
    }
    et en employant les propriétés des limites,
    on obtient
    \begin{align}
        \lim_{x \to a} f(x)
        &= \left(\lim_{x \to a} \frac {f(x) - f(a)} {x - a}\right) \lim_{x \to a} (x - a) + \lim_{x \to a} f(a)\\
        &= f'(a) \cdot 0 + f(a) = f(a).
    \end{align}
\end{proof}

Le choix de mesurer les angles en radian
(c'est-à-dire une unité où le tour complet est représenté par un nombre \emph{irrationnel}\footnote{%
    Formulé comme ça, ça a quand même l'air d'être une décision vachement stupide.
    En plus, $\pi$ est transcendant, ce qui est le pire type de nombre irrationnel.

    En tant qu'ami de Monsieur Masson (paix à son âme),
    je me vois également moralement obligé de mentionner
    que la valeur de $\pi$ devrait être doublée pour correspondre au tour complet.
})
est justifié par la proposition suivante (admise sans preuve).

\begin{proposition}
    [Dérivée des fonctions trigonométriques en radian]

    Pour tout $x \in \R$,
    nous avons
    \begin{align}
        \sin' x &= \cos x\\
        \cos' x &= -\sin x.
    \end{align}
\end{proposition}

\section{Comportement avec les opérations sur les fonctions}

Une dérivée peut se calculer terme par terme.

\begin{proposition}
    [Dérivée d'une somme]

    Soient $f, g$ deux fonctions dérivables en $a \in \R$.
    Alors $f + g$ est dérivable en $a$ et
    \begin{align}
        (f + g)'(a) = f'(a) + g'(a).
    \end{align}
\end{proposition}
\begin{proof}
    Soit $x \in \dom f \setminus \{a\}$.
    Le quotient différentiel de $f + g$ s'écrit
    \begin{align}
        \frac {(f + g)(x) - (f + g)(a)} {x - a}
        &= \frac {f(x) - f(a)} {x - a} + \frac {g(x) - g(a)} {x - a}.
    \end{align}

    En faisant tendre $x$ vers $a$ des deux côtés,
    on obtient
    \begin{align}
        (f + g)'(a) = f'(a) + g'(a).
    \end{align}
\end{proof}

\begin{proposition}
    [Dérivée d'une différence]

    Soient $f, g$ deux fonctions dérivables en $a \in \R$.
    Alors $f - g$ est dérivable en $a$ et
    \begin{align}
        (f - g)'(a) = f'(a) - g'(a).
    \end{align}
\end{proposition}

\begin{proposition}
    [Dérivée d'un produit]

    Soient $f, g$ deux fonctions dérivables en $a \in \R$.
    Alors $f g$ est dérivable en $x$ et
    \begin{align}
        (f g)'(a) = f'(a) g(a) + f(a) g'(a).
    \end{align}
\end{proposition}
\begin{proof}
    Soit $x \in \dom f \setminus \{a\}$.
    Le quotient différentiel de $f g$ s'écrit
    \begin{align}
        \frac {(f g)(x) - (f g)(a)} {x - a}
        &= \frac {f(x) g(x) - f(a) g(a)} {x - a}\\
        &= \frac {f(x) g(x) - f(a) g(x) + f(a) g(x) - f(a) g(a)} {x - a}\\
        &= \left(\frac {f(x) - f(a)} {x - a}\right) g(x) + f(a) \left(\frac {g(x) - g(a)} {x - a}\right).
    \end{align}

    En faisant tendre $x$ vers $a$ des deux côtés,
    on obtient
    \begin{align}
        (f g)'(a) = f'(a) g(a) + f(a) g'(a).
    \end{align}
\end{proof}

\begin{proposition}
    [Dérivée d'une composée]

    \subsubsection{Hypothèses}

    \begin{enumerate}
        \item $f$ une fonction dérivable en $a$;
        \item $g$ une fonction dérivable en $f(a)$.
    \end{enumerate}

    \subsubsection{Thèse}

    Alors $g \circ f$ est dérivable en $a$ et
    \begin{align}
        (g \circ f)'(a) = g'(f(a)) f'(a).
    \end{align}
\end{proposition}
\begin{proof}[Preuve lorsque $f$ est injective]
    Soit $x \in \dom f \setminus \{a\}$.
    Le quotient différentiel de $g \circ f$ s'écrit
    \begin{align}
        \frac {(g \circ f)(x) - (g \circ f)(a)} {x - a}
        &= \frac {g(f(x)) - g(f(a))} {x - a}\\
        &= \frac {g(f(x)) - g(f(a))} {f(x) - f(a)} \frac {f(x) - f(a)} {x - a}.
    \end{align}
    Notons que l'hypothèse d'injectivité nous assure qu'il n'y a pas de division par $0$.

    En faisant tendre $x$ vers $a$ des deux côtés,
    et en se souvenant qu'alors $f(x)$ tend vers $f(a)$ par continuité,
    on obtient
    \begin{align}
        (g \circ f)'(a) = g'(f(a)) f'(a).
    \end{align}
\end{proof}

\begin{proposition}
    [Dérivée d'un quotient]

    Soient $f, g$ deux fonctions dérivables en $a \in \R$.
    Si $g'(a) \neq 0$,
    alors $\frac f g$ est dérivable en $a$ et
    \begin{align}
        \left(\frac f g\right)'(a) = \frac {f'(a) g(a) - f(a) g'(a)} {g^2(a)}
    \end{align}
\end{proposition}
\begin{proof}
    Soit $x \in \dom f \setminus \{a\}$.
    Clairement,
    on a
    \begin{align}
        \left(\frac f g\right)'(a)
        = \left(f \frac 1 g\right)'(a)
        = f'(a) \frac 1 {g(a)} + f(a) \left(\frac 1 g\right)'(a).
    \end{align}

    Il ne reste plus qu'à calculer la dérivée de $\frac 1 g$.
    Pour ce faire,
    si nous dénotons par $\iota(a) = \frac 1 a$ la fonction inverse
    alors $\frac 1 g = \iota \circ g$ et la dérivée vaut
    \begin{align}
        \left(\frac 1 g\right)'(a) = \iota'(g(a)) g'(a) = \frac {-g'(a)} {g^2(a)}.
    \end{align}

    En combinant tout,
    nous obtenons
    \begin{align}
        \left(\frac f g\right)'(a)
        &= f'(a) \frac 1 {g(a)} - f(a) \frac {g'(a)} {g^2(a)}\\
        &= \frac {f'(a) g(a) - f(a) g'(a)} {g^2(a)}
    \end{align}
\end{proof}

\begin{exercise}
    Calculer la dérivée des fonctions tangente et cotangente.
\end{exercise}

\section{Dérivée et croissance}

Rappelons qu'une fonction $f$ est croissante (resp.\ décroissante) sur un intervalle $I$
si le \emph{taux d'accroissement} entre deux points distincts $a \in I$ et $b \in I$
\begin{align}
    \frac {f(b) - f(a)} {b - a}
\end{align}
est toujours \emph{positif} (resp.\ \emph{négatif}).

La \emph{croissance} est liée à la dérivée via le théorème suivant.

\begin{proposition}
    [Théorème des accroissements finis]

    Soit $f$ une fonction dérivable sur un intervalle $\ccinterval a b$, avec $a < b$.
    Il existe $c \in \ccinterval a b$ tel que
    \begin{align}
        \frac {f(b) - f(a)} {b - a} = f'(c).
    \end{align}
\end{proposition}
\begin{proof}[Ébauche de preuve]
    Définissons la fonction
    \begin{align}
        g(x) = f(x) - \frac {f(b) - f(a)} {b - a} (x - a).
    \end{align}

    On vérifie que $g(a) = g(b)$ par choix de $g$.
    Puisque $g$ atteint un extremum sur $\ccinterval a b$,
    disons en $c \in \ccinterval a b$,
    et donc on a $g'(c) = 0$ par le théorème de Fermat.

    Cependant,
    \begin{align}
        g'(x) = f'(x) - \frac {f(b) - f(a)} {b - a}
    \end{align}
    de telle sorte que $g'(c) = 0$ s'écrit
    \begin{align}
        0 = f'(c) - \frac {f(b) - f(a)} {b - a}.
    \end{align}
\end{proof}

\begin{proposition}
    Soit $f$ une fonction dérivable sur un intervalle $I$.
    Alors $f$ est croissante (resp.\ décroissante) sur $I$
    si et seulement si
    $f'$ est positive (resp.\ négative) sur $I$.
\end{proposition}

\begin{howto}
    [Étude de croissance]

    \begin{enumerate}
        \item Conditions d'existence et calcul du domaine.
        \item Calcul de la dérivée.
        \item Tableau de signe de la dérivée.
        \item En déduire la croissance.
    \end{enumerate}
\end{howto}

\subsection{Recherche d'extrémas locaux}

\begin{proposition}
    [Théorème de Fermat]

    Soit $f$ une fonction dérivable sur un intervalle \emph{ouvert} $I$.
    Alors $f$ admet un extremum local en $a \in I$
    si et seulement si
    $f'$ change de signe en $a$\footnote{%
        En rappelant que $0$ est à la fois positif et négatif.
    }.
\end{proposition}

\section{Dérivée seconde, courbure et concavité}

La \emph{dérivée seconde} d'une fonction $f$,
notée $f''$,
est la dérivée de $f'$.

\begin{definition}
    [Convexité]

    Soit $f$ une fonction deux fois dérivable.
    On dit que $f$ est \emph{convexe} (resp.\ \emph{concave}) sur un intervalle $I \subset \dom f$
    si $f''$ est \emph{positive} (resp.\ \emph{négative}) sur $I$.
\end{definition}

\end{document}
