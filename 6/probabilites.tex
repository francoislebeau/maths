\documentclass[main.tex]{subfiles}
\begin{document}

\onlyinsubfile{%
    \maketitle
    \section*{Contributeurs}

\begin{itemize}
    \item David Bertrand, mathématicien, enseignant au lycée Maria Assumpta.
    \item François Lebeau, économiste.
    \item Khoi Nguyen, mathématicien, enseignant au Collège du Christ-Roi.
    \item Urbain Vaes, ingénieur civil, doctorant à Imperial College London.
\end{itemize}

    \tableofcontents
}

\chapter{Probabilités}

\section{Introduction}

\section{Probabilité discrète}

\begin{definition}
    [Expérience aléatoire]

    Une \emph{expérience aléatoire} est une expérience
    dont on ne peut prédire l'issue finale.
\end{definition}

\begin{definition}
    [Probabilité (discrète)]

    \subsubsection{Cadre}
    \begin{itemize}
        \item $\Omega$ désigne toutes les issues possible d'une expérience aléatoire.
    \end{itemize}

    \subsubsection{Définition}
    Une probabilité est une fonction qui à chaque \emph{issue} $I \in \Omega$
    associe un nombre $\Pr(I)$ dans $\ccinterval 0 1$ de telle sorte que
    \begin{align}
        \sum_{I \in \Omega} \Pr(I) = 1
    \end{align}
\end{definition}

\begin{definition}
    [Probabilité d'un événement]

    \subsubsection{Cadre}
    \begin{itemize}
        \item $\Omega$ désigne toutes les issues possible d'une expérience aléatoire.
        \item $E \subset \Omega$ est un événement.
    \end{itemize}

    \subsubsection{Cadre}
    La \emph{probabilité} de $E$,
    notée $\Pr(E)$,
    est la somme des probabilités des issues qui composent $E$.
    Autrement dit,
    on a
    \begin{align}
        \Pr(E) \defeq \sum_{I \in E} \Pr(I).
    \end{align}
\end{definition}

Les programmes requièrent que l'on appelle la propriété suivante \emph{Axiomes de Kolmogorov}.
À chaque fois qu'un enseignant de mathématiques se plie à cette demande,
Kolmogorov et Lebesgue se retournent dans leur tombe,
et les français s'insurgent qu'on oublie une de leurs (deux) contributions du siècle dernier.
Nous arrivons donc au compromis suivant
(qui peut passer pour une faute de frappe tout à fait innocente auprès de l'inspecteur).

\begin{proposition}
    [Axiomes de olmogorov]

    \begin{enumerate}
        \item Pour tout $E \subset \Omega$,
            on a
            \begin{align}
                0 \leq \Pr(E) \leq 1.
            \end{align}
        \item La probabilité d'$\Omega$ vaut $1$
            \begin{align}
                \Pr(\Omega) = 1.
            \end{align}
        \item Si $A, B \subset \Omega$ sont \emph{disjoints} ($A \cap B = \emptyset$),
            alors
            \begin{align}
                \Pr (A \cup B) = \Pr(A) + \Pr(B)
            \end{align}
    \end{enumerate}
\end{proposition}

\begin{remark}
    [Axiomes de Kolmogorov]

    Essentiellement, pour que les axiomes de Kolmogorov soient complets,
    il manque que si $E_1 \subset E_2 \subset \ldots$ est une suite croissante d'événements,
    alors
    \begin{align}
        \Pr(E) = \lim_{n \to \infty} \Pr(E_n)
    \end{align}
    où $E$ est l'ensemble des issues contenues dans au moins un des $E_n$.

    La différence entre la parodie ci-dessus et les vrais axiomes de Kolmogorov est formellement difficilement perceptible,
    mais conceptuellement très profonde.
    Cette différence est en fait l'une des plus grandes découvertes mathématiques du XX\textsuperscript{e} siècle.

    Elle introduit énormément de complexité en théorie des probabilités
    (axiome du choix, tribus)
    mais lui donne également ses plus beaux résultats
    (théorème central limite, loi forte des grands nombres).
\end{remark}

\section{Équiprobabilité et analyse combinatoire}

\begin{definition}
    [Équiprobabilité]

    Une expérience aléatoire est dite \emph{équiprobable}
    si chacune des issues a la \emph{même probabilité}.
\end{definition}

\begin{proposition}
    [Probabilités d'une expérience équiprobable]

    \subsubsection{Hypothèses}
    \begin{itemize}
        \item On conduit une expérience aléatoire \emph{équiprobable}.
        \item $\Omega$ désigne toutes les issues possibles de cette expérience.
        \item $E \subset \Omega$ est un événement.
    \end{itemize}

    \subsubsection{Thèse}
    La probabilité de $E$ est donnée par
    \begin{align}
        \Pr(E) = \frac {\text{nombre de cas favorables}} {\text{nombre de cas possibles}}
        = \frac {\card(E)} {\card(\Omega)}.
    \end{align}
\end{proposition}

\subsection{Espérance}

\begin{definition}
    [Espérance (cas équiprobable)]

    \subsubsection{Cadre}
    \begin{itemize}
        \item On conduit une expérience aléatoire \emph{équiprobable}.
        \item $\Omega$ désigne toutes les $n$ issues possibles de cette expérience.
        \item $X$ est une variable aléatoire sur $\Omega$.
    \end{itemize}

    \subsubsection{Définition}
    La moyenne des valeurs atteintes par $X$ sur $\Omega$,
    \begin{align}
        \E(X) \defeq \frac 1 n \sum_{I \in \Omega} X(I)
    \end{align}
    est appelée l'\emph{espérance} de $X$.
\end{definition}

\begin{definition}
    [Espérance]

    \subsubsection{Cadre}
    \begin{itemize}
        \item On conduit une expérience aléatoire \emph{équiprobable}.
        \item $\Omega$ désigne toutes les $n$ issues possibles de cette expérience.
        \item $X$ est une variable aléatoire sur $\Omega$.
    \end{itemize}

    \subsubsection{Définition}
    La moyenne des valeurs atteintes par $X$ sur $\Omega$,
    \begin{align}
        \E(X) \defeq \sum_{x_i \in \im(X)} x_i \Pr(X = x_i).
    \end{align}
    est appelée l'\emph{espérance} de $X$.
\end{definition}

\end{document}
