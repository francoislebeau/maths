\documentclass{scrartcl}
\usepackage{../common}
\title{Limites et asymptotes}
\begin{document}
\maketitle

\section{Introduction}

% TODO David: À retravailler
De nombreuses notions sont introduites dans le secondaire inférieur par l'intermédiaire d'expressions
qui ne sont valides que dans des cas particuliers.
Pour ne citer qu'elle,
la vitesse est introduite comme étant le quotient entre un déplacement et un intervalle de temps.
Le cas du mouvement rectiligne uniforme mis à part,
cette définition n'est malheureusement qu'approximative
mais devient de plus en plus exacte
au fur et à mesure que l'intervalle de temps considéré devient petit.
Dans cette optique,
il serait tentant de calculer le quotient sur un intervalle de temps nul,
mais nous sortons malheureusement de son domaine de définition.

De nombreuses quantités physiques rencontrent le même problème.
Pour les traiter toutes simultanément,
nous identifions le problème abstrait sous-jacent suivant:
peut-on généraliser la notion d'image d'un point en dehors du domaine?
C'est pour répondre à cette question fondamentale que le concept de \emph{limite} est introduit.

Les applications des limites sont si vastes
que ce sont les mathématiques qui se sont accomodées pour correctement les intégrer.
C'est en effet pour cette raison que les nombres réels sont introduits.
De même,
une volonté d'obtenir des limites plus esthétiques a forcé la disparition du degré comme mesure d'angle privilégiée au profit du radian.

Plus simplement,
la notion de limite est fondamentale
car elle constitue le pilier sur lequel repose le \emph{calcul différentiel et intégral}.

\end{document}
